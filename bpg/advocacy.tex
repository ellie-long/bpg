% vvvvvvvvvvvvvvvvvvvvvvvvvvvvvvvvvvvvvvvvvvvvvvvvvvvvvvvvvvvvvvvvvvvvvvvvvvvvvvvv
% Advocacy at the University Level (advocacy.tex)
%
% Chapter 4 of LGBTPhys-Org Best Practices Guide
% ^^^^^^^^^^^^^^^^^^^^^^^^^^^^^^^^^^^^^^^^^^^^^^^^^^^^^^^^^^^^^^^^^^^^^^^^^^^^^^^^

\chapter{Advocacy at the University Level}	% Chapter Title
\label{univ-advocacy}		% Chapter Label
\normalsize			% Return to Normal font size


\section {Become an advocate}
\label{become-advocate}
% vvvvvvvvvvvvvvvvvvvvvvvvvvvvvvvvvvvvvvvvvvvvvvvvvvvvvvvvvvvvvvvvvvvvvvvvvvvvvvv
If your institution's policies are not inclusive, lobby to change them.  Advocate for inclusion of the words ``sexual orientation'' and ``gender identity or gender expression'' in your institution's non-discrimination policy.  (See Appendix \ref{resources} for resources.)  If you lose existing or prospective faculty, staff or students because of institutional policies that are unwelcoming to LGBT+ persons, notify your top administrators -- especially if departing faculty members take grant money with them.


\section {Use gender-neutral and inclusive language}
\label{univ-language}
% vvvvvvvvvvvvvvvvvvvvvvvvvvvvvvvvvvvvvvvvvvvvvvvvvvvvvvvvvvvvvvvvvvvvvvvvvvvvvvv
Encourage your institution to include ``sexual orientation'' and ``gender identity and gender expression'' in its public statements about diversity and multiculturalism.  Include LGBT+ issues and concerns in grievance procedures, housing guidelines, application materials, health-care forms, and alumni materials and publications.  Include representations of LGBT+ people in these publications.  Until a university-wide policy is in place, your department can take an active role in improving the language that is used at your institution by pointing out instances of discrimination through language and work with other departments (including human resources, benefits, and public relations) to adopt gender-neutral and inclusive wording.
% ^^^^^^^^^^^^^^^^^^^^^^^^^^^^^^^^^^^^^^^^^^^^^^^^^^^^^^^^^^^^^^^^^^^^^^^^^^^^^^^


\section {Provide restroom accessibility}
\label{restrooms}
% vvvvvvvvvvvvvvvvvvvvvvvvvvvvvvvvvvvvvvvvvvvvvvvvvvvvvvvvvvvvvvvvvvvvvvvvvvvvvvv
For many people, especially transgender, intersex or gender-nonconforming people, the availability of restroom space is a frequent and stressful concern. It is important to be able to use the restroom in peace, without being harassed or interrogated about whether it's the ``right" one. To mitigate this worry, express a clear policy that students, faculty and staff may use any restrooms appropriate to their gender identities, and designate some restrooms as all-gender. Usually, all-gender restrooms are single-stall; new ones may be added in a building renovation or existing ones may be re-labeled with an inclusive sign. These restrooms provide critical infrastructure for people with disabilities, family needs, and people with privacy or medical concerns.

Some universities, such as American University and Kent State University, now require newly built and significantly renovated structures to include at least one all-gender restroom each. These restrooms have signs that only read ``Restroom" and/or contain both male and female symbols. Elsewhere, individual departments may keep this concern in mind when contemplating changes to the department's space.
% ^^^^^^^^^^^^^^^^^^^^^^^^^^^^^^^^^^^^^^^^^^^^^^^^^^^^^^^^^^^^^^^^^^^^^^^^^^^^^^^


\section {Identify your LGBT+ students}
\label{univ-identify}
% vvvvvvvvvvvvvvvvvvvvvvvvvvvvvvvvvvvvvvvvvvvvvvvvvvvvvvvvvvvvvvvvvvvvvvvvvvvvvvv
In 2013, the {\em New York Times} reported that a small but growing number of colleges is including questions about sexual orientation and gender identity in their undergraduate applications \footnote{\link{http://www.nytimes.com/2013/08/04/education/edlife/more-college-applications-ask-about-sexual-identity.html?_r=0}{http://www.nytimes.com/2013/08/04/education/edlife/more-college-applications-ask-about-sexual-identity.html?\_r=0}}.  The principal goal is to make prospective LGBT+ students feel welcome, but the information allows schools to consider these students for scholarships aimed at increasing diversity on campus, to provide them with information about support services for LGBT+ students, and to track their success relative to their heterosexual and cisgender peers.  Protecting students' privacy is key, so some schools do not transfer this information to the student's permanent file.  Still, you cannot know how well you are serving your LGBT+ students until you know who they are.
% ^^^^^^^^^^^^^^^^^^^^^^^^^^^^^^^^^^^^^^^^^^^^^^^^^^^^^^^^^^^^^^^^^^^^^^^^^^^^^^^


\section {Help trans students deal with Selective Service}
\label{univ-trans}
% vvvvvvvvvvvvvvvvvvvvvvvvvvvvvvvvvvvvvvvvvvvvvvvvvvvvvvvvvvvvvvvvvvvvvvvvvv
People who were assigned male at birth are required to register with the Selective Service System within thirty days of their eighteenth birthday.  This includes persons who transitioned before or since then. People who were assigned female at birth are not required to register regardless of their current gender or transition status.  These requirements are important for all college students, because many government benefits, including federal financial aid and federal employment, are contingent upon Selective Service registration, but they can be particularly daunting for trans students.  The National Center for Transgender Equality has prepared a helpful document\footnote{\link{http://transequality.org/Resources/Selective_Service_only.pdf}{http://transequality.org/Resources/Selective\_Service\_only.pdf}} for trans people dealing with the Selective Service System.  Universities should be aware that trans students may difficulty providing proof of registration.
% ^^^^^^^^^^^^^^^^^^^^^^^^^^^^^^^^^^^^^^^^^^^^^^^^^^^^^^^^^^^^^^^^^^^^^^^^^^^^^^^


\section {Provide inclusive health insurance}
\label{health-insurance}
% vvvvvvvvvvvvvvvvvvvvvvvvvvvvvvvvvvvvvvvvvvvvvvvvvvvvvvvvvvvvvvvvvvvvvvvvvvvvvvv
LGBT+ students, faculty and staff are often unable to take full advantage of university or college health benefits.  Although the Human Rights Campaign reports that 62\% of Fortune 500 companies now extend coverage to domestic partners of their employees\footnote{\link{http://www.hrc.org/resources/entry/lgbt-equality-at-the-fortune-500}{http://www.hrc.org/resources/entry/lgbt-equality-at-the-fortune-500}}, many colleges and universities do not.  In states where health coverage for domestic partners is permitted, advocate with the university administration to allow domestic partners, regardless of gender, to receive coverage. If state law prohibits such an arrangement, it may be possible to pursue an alternative insurance structure subsidized by sympathetic donors. 

Many health plans exclude ``procedures related to being transgender''. As documented by the Transgender Law Center, this exclusion not only applies to medical services that are vital for the transitioning process -- such as hormone treatments, transition surgery, or therapy for those who require it -- but also has been used to deny treatment for pathologies associated with the sex assigned at birth (e.g., uterine cancer in a transgender man) and for such non-gendered problems as the flu or a broken arm. The impact of such policies on an individual's physical and mental health cannot be overstated. Advocate for removing the transgender exclusion from your college or university's health plan, as has been done successfully by the University of California System.  As a smaller step, advocate for including specific coverage for certain procedures (such as therapy, hormone treatments, and sexual reassignment surgery) for transgender students, faculty, staff and family members.

Offering inclusive health insurance is a competitive advantage in hiring candidates who are LGBT+ or who have LGBT+ beneficiaries.

%This is an example of language for defining domestic partners for the purpose of benefits, used by the Thomas Jefferson National Accelerator Facility: ``a non-married cohabiting couple, sharing a common legal residence, for a minimum of twelve months prior; and intending to cohabit indefinitely, sharing the common necessities of life." \footnote{\href{https://www.jlab.org/div_dept/admin/HR/benefits/forms2008/DomesticPartnerAffidavit.pdf}{https://www.jlab.org/div\_dept/admin/HR/benefits/forms2008/\\DomesticPartnerAffidavit.pdf}}
% ^^^^^^^^^^^^^^^^^^^^^^^^^^^^^^^^^^^^^^^^^^^^^^^^^^^^^^^^^^^^^^^^^^^^^^^^^^^^^^^


\section {Provide other benefits fairly}
\label{other-benefits}
% vvvvvvvvvvvvvvvvvvvvvvvvvvvvvvvvvvvvvvvvvvvvvvvvvvvvvvvvvvvvvvvvvvvvvvvvvvvvvvv
Offer other benefits equally to both opposite-sex and same-sex spouses/partners of employees.  These benefits may include access to the gym, the library, and the university credit union.  If child care is provided on campus, make it easy for both parents to take their child home.  

If your state recognizes same-sex marriage, implement policies to support same-sex couples who have recently arrived from less-friendly territories.  The University of Vermont has an innovative program that extends for sixty days (just enough time to plan a wedding) all university benefits to the same-sex domestic partner of a newly-hired employee who comes from a state where civil unions and same-sex marriage are not recognized \footnote{\link{http://usnh.edu/hr/benefits/pdf/2009-11_Hardship_Exception_DomesticPartnerBenefit_Eligibility.pdf}{http://usnh.edu/hr/benefits/pdf/\newline2009-11\_Hardship\_Exception\_DomesticPartnerBenefit\_Eligibility.pdf}}.  
% ^^^^^^^^^^^^^^^^^^^^^^^^^^^^^^^^^^^^^^^^^^^^^^^^^^^^^^^^^^^^^^^^^^^^^^^^^^^^^^^

\section {Appoint a contact person for dual-career couples}
\label{dual-career-contact}
% vvvvvvvvvvvvvvvvvvvvvvvvvvvvvvvvvvvvvvvvvvvvvvvvvvvvvvvvvvvvvvvvvvvvvvvvvvvvvvv
For any potential employee, it may be important to be able to acquire information about dual-career resources in confidence and without affecting the search.  As noted earlier, the issues can be even more complex for same-sex dual-career couples. The university should provide means for job candidates to acquire this information as early as possible in the process, to ensure that there is clarity about the prospects when an offer of employment is made.  Having a visible, comprehensive web page where resources related to dual-career issues and work-life balance are collected is helpful.  Best of all is for the university to appoint a {single contact person for all dual-career issues, regardless of gender}. This person (e.g., a vice-provost for academic human resources), who is far removed from the search committee, can assist in assessing the situation while the hiring process is on-going.  Having a well-informed person from whom potential employees may obtain advice about the details of a dual-career situation can help make all job candidates feel welcome.
% ^^^^^^^^^^^^^^^^^^^^^^^^^^^^^^^^^^^^^^^^^^^^^^^^^^^^^^^^^^^^^^^^^^^^^^^^^^^^^^^


\section {Participate in surveys exploring LGBT+ experiences}
\label{univ-surveys}
% vvvvvvvvvvvvvvvvvvvvvvvvvvvvvvvvvvvvvvvvvvvvvvvvvvvvvvvvvvvvvvvvvvvvvvvvvvvvvvv
Encourage participation in {national or regional surveys} that address LGBT+ issues. For example, Campus Pride produces the LGBT-Friendly Campus Climate Index \link{http://www.campusprideindex.org}{(http://www.campusprideindex.org)}, a valuable resource for students and administrators. An official authorized to represent the college or university on LGBT+ issues may contact Campus Pride to take part in the assessments for the Index.
% ^^^^^^^^^^^^^^^^^^^^^^^^^^^^^^^^^^^^^^^^^^^^^^^^^^^^^^^^^^^^^^^^^^^^^^^^^^^^^^^


















