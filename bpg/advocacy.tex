% vvvvvvvvvvvvvvvvvvvvvvvvvvvvvvvvvvvvvvvvvvvvvvvvvvvvvvvvvvvvvvvvvvvvvvvvvvvvvvvv
% Advocacy at the University Level (advocacy.tex)
%
% Chapter 4 of LGBTPhys-Org Best Practices Guide
% ^^^^^^^^^^^^^^^^^^^^^^^^^^^^^^^^^^^^^^^^^^^^^^^^^^^^^^^^^^^^^^^^^^^^^^^^^^^^^^^^

\chapter{Advocacy at the University Level}	% Chapter Title
\label{univ-advocacy}		% Chapter Label
\normalsize			% Return to Normal font size


\section {Become an advocate}
\label{become-advocate}
% vvvvvvvvvvvvvvvvvvvvvvvvvvvvvvvvvvvvvvvvvvvvvvvvvvvvvvvvvvvvvvvvvvvvvvvvvvvvvvv
If your institution's policies are not inclusive, lobby to change them.  Advocate for inclusion of the words ``sexual orientation'' and ``gender identity or expression'' in your institution's non-discrimination policy.  (See Appendix \ref{resources} for resources.)  If you lose existing or prospective faculty, staff or students because of institutional policies that are unwelcoming to LGBT+ persons, notify your top administrators -- especially if departing faculty members take grant money with them.


\section {Use gender-neutral and inclusive language}
\label{univ-language}
% vvvvvvvvvvvvvvvvvvvvvvvvvvvvvvvvvvvvvvvvvvvvvvvvvvvvvvvvvvvvvvvvvvvvvvvvvvvvvvv
Encourage your institution to include ``sexual orientation'' and ``gender identity/expression'' in its public statements about diversity and multiculturalism.  Include LGBT+ issues and concerns in grievance procedures, housing guidelines, application materials, health-care forms, and alumni materials and publications.  Include representations of LGBT+ people in these publications.  Until a university-wide policy is in place, your department can take an active role in improving the language that is used at your institution by pointing out instances of discrimination through language and work with other departments (including human resources, benefits, and public relations) to adopt gender-neutral and inclusive wording.
% ^^^^^^^^^^^^^^^^^^^^^^^^^^^^^^^^^^^^^^^^^^^^^^^^^^^^^^^^^^^^^^^^^^^^^^^^^^^^^^^


\section {Provide inclusive health insurance}
\label{health-insurance}
% vvvvvvvvvvvvvvvvvvvvvvvvvvvvvvvvvvvvvvvvvvvvvvvvvvvvvvvvvvvvvvvvvvvvvvvvvvvvvvv
LGBT+ students, faculty and staff are often unable to take full advantage of university or college health benefits.  Although the Human Rights Campaign reports that 62\% of Fortune 500 companies now extend coverage to domestic partners of their employees\footnote{\href{http://www.hrc.org/resources/entry/lgbt-equality-at-the-fortune-500}{hrc.org/resources/entry/lgbt-equality-at-the-fortune-500}}, many colleges and universities do not.  In states where health coverage for domestic partners is permitted, \textbf{advocate with the administration to include domestic partners}, regardless of gender, as possible insurance beneficiaries. If state law prohibits such an arrangement, it may be possible to pursue an alternative insurance structure subsidized by sympathetic donors. 

Many health plans exclude ``procedures related to being transgender''. As documented by the Transgender Law Center, this exclusion not only applies to medical services that are vital for the transitioning process -- such as hormone treatments, transition surgery, or therapy for those who require it -- but also has been used to deny treatment for pathologies associated with the sex assigned at birth (e.g., uterine cancer in a transgender man) and for such non-gendered problems as the flu or a broken arm. The impact of such policies on an individual's physical and mental health cannot be overstated. Advocate for removing the transgender exclusion from your college or university's health plan, as has been done successfully by the University of California System.  As a smaller step, advocate for including specific coverage for certain procedures (such as therapy, hormone treatments, and sexual reassignment surgery) for transgender students, faculty, staff and family members.

Offering inclusive health insurance is a competitive advantage in hiring candidates who are LGBT+ or who have LGBT+ beneficiaries.

%This is an example of language for defining domestic partners for the purpose of benefits, used by the Thomas Jefferson National Accelerator Facility: ``a non-married cohabiting couple, sharing a common legal residence, for a minimum of twelve months prior; and intending to cohabit indefinitely, sharing the common necessities of life." \footnote{\href{https://www.jlab.org/div_dept/admin/HR/benefits/forms2008/DomesticPartnerAffidavit.pdf}{https://www.jlab.org/div\_dept/admin/HR/benefits/forms2008/\\DomesticPartnerAffidavit.pdf}}
% ^^^^^^^^^^^^^^^^^^^^^^^^^^^^^^^^^^^^^^^^^^^^^^^^^^^^^^^^^^^^^^^^^^^^^^^^^^^^^^^


\section {Provide other benefits fairly}
\label{other-benefits}
% vvvvvvvvvvvvvvvvvvvvvvvvvvvvvvvvvvvvvvvvvvvvvvvvvvvvvvvvvvvvvvvvvvvvvvvvvvvvvvv
Offer other benefits equally to both opposite-sex and same-sex spouses/partners of employees.  These benefits may include access to the gym, the library, and the university credit union.  If child care is provided on campus, make it easy for both parents to take their child home.  
% ^^^^^^^^^^^^^^^^^^^^^^^^^^^^^^^^^^^^^^^^^^^^^^^^^^^^^^^^^^^^^^^^^^^^^^^^^^^^^^^


\section {Provide restroom accessibility}
\label{restrooms}
% vvvvvvvvvvvvvvvvvvvvvvvvvvvvvvvvvvvvvvvvvvvvvvvvvvvvvvvvvvvvvvvvvvvvvvvvvvvvvvv
For many people, especially transgender, intersex or gender-nonconforming people, the availability of restroom space is a frequent and stressful concern. It is important to be able to use the restroom in peace, without being harassed or interrogated about whether it's the ``right" one. To mitigate this worry, express a clear policy that students, faculty and staff may use any restrooms appropriate to their gender identities, and designate some restrooms as gender-neutral. Usually, gender-neutral restrooms are single-stall; new ones may be added in a building renovation or existing ones may be re-labeled with an inclusive sign. These restrooms provide critical infrastructure for people with disabilities, family needs, and people with privacy or medical concerns.

Some universities, such as American University and Kent State University, now require newly built and significantly renovated structures to include at least one gender-neutral restroom each. These restrooms have signs that only read ``Restroom" and/or contain both male and female symbols. Elsewhere, individual departments may keep this concern in mind when contemplating changes to the department's space.
% ^^^^^^^^^^^^^^^^^^^^^^^^^^^^^^^^^^^^^^^^^^^^^^^^^^^^^^^^^^^^^^^^^^^^^^^^^^^^^^^

\section {Appoint a contact person for dual-career couples}
\label{dual-career-contact}
% vvvvvvvvvvvvvvvvvvvvvvvvvvvvvvvvvvvvvvvvvvvvvvvvvvvvvvvvvvvvvvvvvvvvvvvvvvvvvvv
For any potential employee, it may be important to be able to acquire information about dual-career resources in confidence and without affecting the search.  As noted earlier, the issues can be even more complex for same-sex dual career couples. The university should provide means for job candidates to acquire this information as early as possible in the process, to ensure that there is clarity about the prospects when an offer of employment is made.  Having a visible, comprehensive web page where resources related to dual-career issues and work-life balance are collected is helpful.  Best of all is for the university to appoint a \textbf{single contact person for all dual-career issues, regardless of gender}. This person (e.g., a vice-provost for academic human resources), who is far removed from the search committee, can assist in assessing the situation while the hiring process is on-going.  Having a well-informed person from whom potential employees may obtain advice about the details of a dual-career situation can help make all job candidates feel welcome.
% ^^^^^^^^^^^^^^^^^^^^^^^^^^^^^^^^^^^^^^^^^^^^^^^^^^^^^^^^^^^^^^^^^^^^^^^^^^^^^^^


\section [Participate in surveys exploring LGBT+ experiences]{Participate in surveys exploring LGBT+ experiences}
\label{univ-surveys}
% vvvvvvvvvvvvvvvvvvvvvvvvvvvvvvvvvvvvvvvvvvvvvvvvvvvvvvvvvvvvvvvvvvvvvvvvvvvvvvv
Encourage participation in \textbf{national or regional surveys} that address LGBT+ issues. For example, Campus Pride produces the LGBT-Friendly Campus Climate Index, a valuable resource for students and administrators. An official authorized to represent the college or university on LGBT+ issues may contact Campus Pride to take part in the assessments for the Index.
% ^^^^^^^^^^^^^^^^^^^^^^^^^^^^^^^^^^^^^^^^^^^^^^^^^^^^^^^^^^^^^^^^^^^^^^^^^^^^^^^


















