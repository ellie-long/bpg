% vvvvvvvvvvvvvvvvvvvvvvvvvvvvvvvvvvvvvvvvvvvvvvvvvvvvvvvvvvvvvvvvvvvvvvvvvvvvvvvv
% Resources (resources.tex)
%
% Chapter 5 of LGBTPhys-Org Best Practices Guide
% ^^^^^^^^^^^^^^^^^^^^^^^^^^^^^^^^^^^^^^^^^^^^^^^^^^^^^^^^^^^^^^^^^^^^^^^^^^^^^^^^

\chapter{Resources}	% Chapter Title
\label{resources}		% Chapter Label
\normalsize			% Return to Normal font size

\begin{tabular*}{\textwidth}{@{\extracolsep{\fill}}lr}
	\textbf{LGBT+ Physicists} & \href{http://lgbtphysicists.org}{http://lgbtphysicists.org}	
\end{tabular*}
This website was created to collect resources for and address the issues of LGBT+ people in physics. It contains information on joining LGBT+ Physicists, an Out List, current and past events, and links to other resources.

\vspace*{\baselineskip}
\noindent\begin{tabular*}{\textwidth}{@{\extracolsep{\fill}}lr}
	\textbf{GLAAD Media Reference Guide} & \href{http://www.glaad.org/reference}{http://www.glaad.org/reference}	
\end{tabular*}
Although created for journalists, this guide provides information on terminology used with LGBT+ communities as well as a list of current national issues in the United States faced by the community.

\vspace*{\baselineskip}
\noindent\begin{tabular*}{\textwidth}{@{\extracolsep{\fill}}lr}
	\textbf{CampusPride} & \href{http://www.campuspride.org}{http://www.campuspride.org}	
\end{tabular*}
CampusPride is a leading organization in research on LGBT+ people in colleges and universities. They put together the 2010 State of Higher Education for LGBT People, organize LGBT+ job fairs, and compile lists of those colleges and universities that excel in LGBT+ issues.

\vspace*{\baselineskip}
\noindent\begin{tabular*}{\textwidth}{@{\extracolsep{\fill}}lr}
	\textbf{Campus Climate Index} & \href{http://www.campusprideindex.org}{http://www.campusprideindex.org}	
\end{tabular*}
This index was put together by CampusPride and ranks colleges and universities by how friendly they are to LGBT+ students. It contains information on how ranking is done, as well as how you can add your institution to the list.

%\vspace*{\baselineskip} \newpage
%\noindent\begin{tabular*}{\textwidth}{@{\extracolsep{\fill}}lr}
%	\textbf{How Colleges and Universities Can Improve } & \\ 
%	\textbf{Their Environment for TG/TS Students} & \\ 
%\end{tabular*}
%\begin{tabular*}{\textwidth}{@{\extracolsep{\fill}}lr}
%	 & \href{http://ai.eecs.umich.edu/people/conway/TS/College.html}{http://ai.eecs.umich.edu/people/conway/TS/College.html}	
%\end{tabular*}
%This website, put together by transgender pioneer Lynn Conway, maps out specific issues that are faced by transgender students in colleges and universities. It also includes information on how to make your institution more trans friendly.

\vspace*{\baselineskip}\newpage
\noindent\begin{tabular*}{\textwidth}{@{\extracolsep{\fill}}lr}
	\textbf{Gay, Lesbian, and Straight} & \\
	\textbf{Education Network} & \href{http://www.glsen.org}{http://www.glsen.org}	
\end{tabular*}
Although geared for education from K-12, GLSEN's research provides many insights into LGBT+ students including those soon to become college freshmen. Every year, GLSEN puts together a National School Climate Survey. In 2011, GLSEN found that 81.9\% of LGBT+ middle and high school students experienced harassment in the previous year. They’ve also done studies specifically on transgender students, such as their Harsh Realities report, and on LGBT+ people of color, such as their Shared Differences report. All of their publications can be found at \href{http://www.glsen.org/cgi-bin/iowa/all/research/index.html}{http://www.glsen.org/cgi-bin/iowa/all/research/index.html} with both an executive summary and a full report.

\vspace*{\baselineskip}
\noindent\begin{tabular*}{\textwidth}{@{\extracolsep{\fill}}lr}
	\textbf{NOGLSTP} & \href{http://www.noglstp.org}{http://www.noglstp.org}	
\end{tabular*}
The National Organization of Gay and Lesbian Scientists and Technical Professionals, Inc., is a national professional society. It educates STEM communities about the needs of their LGBT+ members and supports lesbian, gay, bisexual, transgender, and queer people in STEM fields, especially via mentoring, networking, and advocacy.

\vspace*{\baselineskip}
\noindent\begin{tabular*}{\textwidth}{@{\extracolsep{\fill}}lr}
	\textbf{oSTEM} & \href{http://www.ostem.org}{http://www.ostem.org}	
\end{tabular*}
Out in Science, Technology, Engineering, and Mathematics is a national society dedicated to the organization and professional development of LGBT students in STEM. The group consists of affiliate chapters throughout the U.S. and is led by a governing board known as oSTEM Incorporated.

\vspace*{\baselineskip}
\noindent\begin{tabular*}{\textwidth}{@{\extracolsep{\fill}}lr}
	\textbf{The TONI Project} & \href{http://transstudents.org}{http://transstudents.org}	
\end{tabular*}
Organized by the National Center for Transgender Equality (NCTE), the TONI Project is a student-oriented space for sharing college and university practices and policies of particular interest to trans students. Campus-by-campus information may be useful to prospective students choosing a school, or to people hoping to improve policies at their own institutions.

\vspace*{\baselineskip}
\noindent\begin{tabular*}{\textwidth}{@{\extracolsep{\fill}}lr}
	\textbf{The Transgender Law and} & \\
	\textbf{Policy Institute} & \href{http://www.transgenderlaw.org}{http://www.transgenderlaw.org}	
\end{tabular*}
This institute is a non-profit organization dedicated to engaging in effective advocacy for transgender people in society. The TLPI brings experts and advocates together to work on law and policy initiatives designed to advance transgender equality. Of particular interest to colleges and universities, this institute has put together a list of policies that affect transgender students and the institutions which implemented them \\(\href{http://www.transgenderlaw.org/college/index.htm}{http://www.transgenderlaw.org/college/index.htm}).



\vspace*{\baselineskip}
\noindent\begin{tabular*}{\textwidth}{@{\extracolsep{\fill}}lr}
	\textbf{Post-DOMA Fact Sheet}  \\
\end{tabular*}
\noindent\begin{tabular*}
{\textwidth}{@{\extracolsep{\fill}}lr}	
	 & \href{http://www.aclu.org/lgbt-rights/after-doma-what-it-means-you}{http://www.aclu.org/lgbt-rights/after-doma-what-it-means-you}	
\end{tabular*}

The American Civil Liberties Union has joined with several other organizations to compile a very useful set of fact sheets, describing the legal situation of married same-sex couples after the US Supreme Court struck down the so-called "Defense of Marriage Act" (DOMA). This decision requires the US federal government to recognize legal same-sex marriages, but does not require individual states to do so, resulting in a confusing patchwork of protections and obligations that depend on the state where your institution is located and the state where the couple was married. The fact sheets are an educational resource for same-sex couples, students whose parents have a same-sex marriage, human resources departments, and other administrators, covering employment benefits, taxes, immigration, federal student aid, and more.


\vspace*{\baselineskip}
\noindent\begin{tabular*}{\textwidth}{@{\extracolsep{\fill}}lr}
	\textbf{Imigration Equality} & \href{http://immigrationequality.org}{http://immigrationequality.org}	
\end{tabular*}
This is a nonprofit advocacy group focused on U.S. immigration issues as applied to members of the LGBT+ community. Their website provides up-to-date resources for issues frequently faced by transgender immigrants, binational same-sex couples, other LGB+ individuals, and people who are HIV-positive. They are also sometimes able to provide legal help or recommend a private attorney.


