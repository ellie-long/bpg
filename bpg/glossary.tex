% vvvvvvvvvvvvvvvvvvvvvvvvvvvvvvvvvvvvvvvvvvvvvvvvvvvvvvvvvvvvvvvvvvvvvvvvvvvvvvvv
% Glossary (glossary.tex)
%
% Glossary of LGBTPhys-Org Best Practices Guide
% ^^^^^^^^^^^^^^^^^^^^^^^^^^^^^^^^^^^^^^^^^^^^^^^^^^^^^^^^^^^^^^^^^^^^^^^^^^^^^^^^
%\frontmatter
%\chapter{Glossary}	% Chapter Title
\begin{titlepage}
\setcounter{page}{7}
\hfill\Large{GLOSSARY}
\label{glossary}		% Chapter Label
\normalsize			% Return to Normal font size
\vspace*{\baselineskip}

\emph{Generally the glossary appears at the end of a book.  We put it at the beginning to illustrate the importance of using a common language to discuss GLBT+ issues. Use of this language will identify you as someone who has thought about these issues and may provoke important conversations with both GLBT+ and non-GLBT+ colleagues.}
\vspace*{\baselineskip}

\emph{Some of the terms in this glossary may be unfamiliar; some familiar terms may have unfamiliar definitions. However, this language has evolved out of the literature and debates on gender and sexual diversity issues over the last few decades. The glossary's definitions are composites of widely used and accepted forms. They can be used as a launching point for the reader to understand the differences that exist between ``day-to-day" use of the terms and more inclusive and explanatory definitions.}
\vspace*{\baselineskip}

<<<<<<< HEAD
%\noindent\textbf{Ally}\\
%Someone who is not part of the LGBT+ community but works to ensure equal rights and opportunities for LGBT+ people.\vspace*{\baselineskip}
%
%\noindent\textbf{Asexuality}\\
%Asexuality (or nonsexuality) is the lack of sexual attraction to anyone or low or absent interest in sexual activity. \vspace*{\baselineskip}

\noindent\textbf{Bisexuality}\\
Sexual orientation characterized by romantic and sexual attraction to both men and women.\vspace*{\baselineskip}

\noindent\textbf{Cisgender or Cis}\\
Term referring to a person who identifies their gender to be in line with their {\em sex}. For example, someone who was assigned female at birth and identifies as a woman is considered to be cisgender.\vspace*{\baselineskip}

\noindent\textbf{Gay}\\
Term referring to a person who is romantically and sexually attracted to people of the same gender.  Formally referring to a man who is attracted to other men, the term can be and is often used by lesbians. \vspace*{\baselineskip}

\noindent\textbf{Gender}\\
Gender refers to the roles, behaviors, activities, and attributes that a society considers appropriate for men and women. It is distinct from {\em sex,} which is a category defined at birth based on physical characteristics. \vspace*{\baselineskip}

\noindent\textbf{Gender Expression}\\
How a person represents or expresses themselves in relation to gender -- through the clothes they wear, their hairstyle, or their mannerisms. A person's gender expression may not always match their {\em sex} or {\em gender identity} and can change from situation to situation or from day to day. \vspace*{\baselineskip}

\noindent\textbf{Gender Identity}\\
An individual's private sense and personal experience of their gender. One's gender identity need not be in line with their {\em sex}. Most people find that their gender identity lies somewhere between the extremes of male and female; some individuals do not identify with either gender. \vspace*{\baselineskip}
=======
\noindent\textbf{Ally}\\
Someone who is not part of the LGBT+ community but works to ensure equal rights and opportunities for LGBT+ people.\vspace*{\baselineskip}

\noindent\textbf{Bisexuality}\\
Sexual orientation characterized by attraction to both men and women.\vspace*{\baselineskip}

\noindent\textbf{Cisgender or Cis}\\
Term referring to a person who identifies their gender to be in line with the sex assigned to them at birth. For example, someone who was assigned female at birth and identifies as a woman is considered to be cisgender.\vspace*{\baselineskip}

\noindent\textbf{Gay}\\
Term referring to a homosexual person, frequently but not exclusively a homosexual man.\vspace*{\baselineskip}

\noindent\textbf{Gender}\\
Roles and identities that are socially constructed for men and women. Gender is not an inherent trait, but rather a fluid identity on the spectrum of socially defined femininity and masculinity.\vspace*{\baselineskip}

\noindent\textbf{Gender Identity}\\
An individual's feeling towards and experience of their personal gender. Gender identity does not have to be in line with the sex assigned at birth. For example, a person assigned female at birth can have a male gender identity.\vspace*{\baselineskip}
>>>>>>> 60d95d952bf9e6111a0047a0b5bd646825775559

\noindent\textbf{Gender Minority}\\
An individual in a situation where their gender is not as widely represented as others. For example, a woman in physics is a gender minority.\vspace*{\baselineskip}

<<<<<<< HEAD
%\noindent\textbf{Genderqueer}\\
%A term sometimes used by people who feel that their {\em gender identity} or {\em gender expression} does not conform to societal expectations. Use with caution as it may be perceived as offensive. As a rule, do not apply this term to others unless they have chosen to explicitly self-identify this way. See also {\em trans}. \vspace*{\baselineskip}

\noindent\textbf{Heterosexuality}\\
Sexual orientation characterized by romantic and sexual attraction to individuals of another gender.\vspace*{\baselineskip}

\noindent\textbf{Homosexuality}\\
Sexual orientation characterized by romantic and sexual attraction to individuals of the same gender.\vspace*{\baselineskip}

\noindent\textbf{Intersex}\\
A {\em sex} assigned at birth for persons exhibiting physical characteristics of both males and females, usually due to variations in prenatal development.\vspace*{\baselineskip}

\noindent\textbf{Lesbian}\\
Term referring to a woman who is romantically and sexually attracted to other women.\vspace*{\baselineskip}
=======
\noindent\textbf{Heterosexuality}\\
Sexual orientation characterized by attraction to members of another gender.\vspace*{\baselineskip}

\noindent\textbf{Homosexuality}\\
Sexual orientation characterized by attraction to members of the same gender.\vspace*{\baselineskip}

\noindent\textbf{Intersex}\\
A sex assigned at birth for persons exhibiting characteristics of both birth-assigned males and females, usually due to variations in prenatal development.\vspace*{\baselineskip}

\noindent\textbf{Lesbian}\\
Term referring to a homosexual woman.\vspace*{\baselineskip}
>>>>>>> 60d95d952bf9e6111a0047a0b5bd646825775559

\noindent\textbf{LGBT+}\\
Lesbian, Gay, Bisexual, Transgender. The plus recognizes that not everyone fits their personal identity neatly into the LGBT constructs and may identify differently.\vspace*{\baselineskip}

<<<<<<< HEAD
\noindent\textbf{Out (of the Closet)}\\
Openly identifying oneself as LGBT+. Someone may be out to some people but not to others (e.g., at school but not to family members, or vice versa). The decision to come out is highly personal. No one should be outed without their explicit prior agreement, as this can be harmful and even dangerous.\vspace*{\baselineskip}

%\noindent\textbf{Pansexual}\\
%An individual who is sexually and/or romantically attracted to members of all gender identities. In most situations, this is interchangeable with bisexual, but avoids the  assumption that gender is binary, instead acknowledging that gender for many people lies on the spectrum between masculine and feminine. \vspace*{\baselineskip}

\noindent\textbf{Preferred Gender Pronoun}\\
The pronoun or set of pronouns that an individual would like others to use when talking to or about that individual. %If you are unsure how someone identifies, simply ask, ``What is your preferred gender pronoun?'' Some people use he/him/his or she/her/hers, while others may prefer that you use gender neutral or gender inclusive pronouns when talking to or about them. In English, the most commonly used singular gender-neutral pronouns are ze (sometimes spelled zie) and hir. ``Ze'' is the subject pronoun and is pronounced /zee/, and ``hir'' is the object and possessive pronoun and is pronounced /heer/. This is how they are used: ``Chris is the tallest person in class, and ze is also the fastest runner.'' ``Tanzen is going to Hawaii over break with hir parents. I�m so jealous of hir.''  
\vspace*{\baselineskip}

\noindent\textbf{Queer}\\
A former term of abuse that has been reclaimed by some members of the LGBT+ community as an identity that may be used in place of or in conjunction with other identities from the LGBT+ spectrum. Like all reclaimed words, it should be used with caution by persons outside of the community.
%It has been adopted by a variety of academic disciplines, such as Queer Theory and Queer Theology, that aim to scrutinize the binary view of gender in history, art, theology, sociology, etc. To some, it may confer a connotation of political engagement with LGBT+ issues. Like all reclaimed words, it is important to be sensitive to those who may take offense when it is used by persons outside of the community.
\vspace*{\baselineskip}

\noindent\textbf{Questioning}\\
An individual who is not yet certain of their sexual orientation or gender identity. 
%Many people in the LGBT+ community go through a stage of questioning their identity during the coming-out process, while others question their identity and eventually identify as straight and cisgender. 
Questioning is considered to be a legitimate identity in itself, and those who identify as such should not be coerced to ``make up their minds.'' \\ %\vspace*{\baselineskip}

\noindent\textbf{Sex}\\
A category, such as male, female, or intersex, assigned at birth based on physical characteristics.\vspace*{\baselineskip}

\noindent\textbf{Sexual Minority}\\
A person in a situation where their sexual orientation is not as widely represented as others. For example, a gay male in physics is a sexual minority. \\ %\vspace*{\baselineskip}

\noindent\textbf{Trans*}\\
An umbrella term used to describe the entire spectrum of people whose {\em gender identity} or {\em gender expression} does not conform to societal expectations or their {\em sex}.  The term has been widely adopted by the community, is safe and respectful to use, and considered fully inclusive.  (The asterisk serves as a sort of wild card.) \vspace*{\baselineskip}

\noindent\textbf{Transgender}\\
An adjective used to describe someone whose {\em gender identity} does not match their {\em sex}. The term has been widely adopted by the community and is safe and respectful to use. %Should only be used as an adjective (i.e. "Mark is transgender") never as a noun (i.e. "Mark is a transgender")
\vspace*{\baselineskip}

%\noindent\textbf{Transsexual}\\
%An outdated term used to describe transgender individuals who have undergone gender reassignment surgery. It is often considered offensive and intrusive, so should not be used unless someone specifically identifies themselves in this way. \vspace*{\baselineskip}
%
%\noindent\textbf{Transvestite}\\
%Related terms: cross-dresser, drag queen, drag king. An individual who chooses to dress in clothes that society associates with a different gender. The term can be considered offensive, so should be used with caution. Individuals who identify as transvestite are not necessarily part of the LGBT community; they may or may not be transgender, and may or may not be LGB. 

=======
%\noindent\textbf{oSTEM}\\
%Out in Science, Technology, Engineering, and Mathematics is a national society dedicated to the organization and professional development of LGBT students in STEM. The group consists of affiliate chapters throughout the U.S. and is led by a governing board known as oSTEM Incorporated.\vspace*{\baselineskip}

\noindent\textbf{Out (of the Closet)}\\
Openly identifying oneself as LGBT+. Someone may be out to some people but not to others (e.g. at school but not to family members, or vice versa). The decision to come out is highly personal. No one should be outed without their explicit prior agreement, as this can be harmful and even dangerous.\vspace*{\baselineskip}

\noindent\textbf{Transgender or Trans}\\
Term referring to a person who identifies their gender with one not in line with the sex assigned to them at birth. For example, a trans woman is someone who was assigned male at birth but whose gender identity is female.\vspace*{\baselineskip}

%\noindent\textbf{WISE}\\
%Women in Science and Engineering, a campus organization.\vspace*{\baselineskip}

\noindent\textbf{Sex}\\
A category, such as male, female, or intersex, assigned at birth based on biological characteristics.\vspace*{\baselineskip}

\noindent\textbf{Sexual Minority}\\
A person in a situation where their sexual orientation is not as widely represented as others. For example, a gay male in physics is a sexual minority.
>>>>>>> 60d95d952bf9e6111a0047a0b5bd646825775559
\end{titlepage}








