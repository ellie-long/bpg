% vvvvvvvvvvvvvvvvvvvvvvvvvvvvvvvvvvvvvvvvvvvvvvvvvvvvvvvvvvvvvvvvvvvvvvvvvvvvvvvv
% Glossary (glossary.tex)
%
% Glossary of LGBTPhys-Org Best Practices Guide
% ^^^^^^^^^^^^^^^^^^^^^^^^^^^^^^^^^^^^^^^^^^^^^^^^^^^^^^^^^^^^^^^^^^^^^^^^^^^^^^^^
%\frontmatter
%\chapter{Glossary}	% Chapter Title
\begin{titlepage}
\setcounter{page}{7}
\hfill\Large{GLOSSARY}
\label{glossary}		% Chapter Label
\normalsize			% Return to Normal font size
\vspace*{\baselineskip}

\emph{Generally the glossary appears at the end of a book.  We put it at the beginning to illustrate the importance of using a common language to discuss GLBT+ issues. Use of this language will identify you as someone who has thought about these issues and may provoke important conversations with both GLBT+ and non-GLBT+ colleagues.}
\vspace*{\baselineskip}

\emph{Some of the terms in this glossary may be unfamiliar; some familiar terms may have unfamiliar definitions. However, this language has evolved out of the literature and debates on gender and sexual diversity issues over the last few decades. The glossary's definitions are composites of widely used and accepted forms. They can be used as a launching point for the reader to understand the differences that exist between ``day-to-day" use of the terms and more inclusive and explanatory definitions.}
\vspace*{\baselineskip}

\noindent\textbf{Ally}\\
Someone who is not part of the LGBT+ community but works to ensure equal rights and opportunities for LGBT+ people.\vspace*{\baselineskip}

\noindent\textbf{Bisexuality}\\
Sexual orientation characterized by attraction to both men and women.\vspace*{\baselineskip}

\noindent\textbf{Cisgender or Cis}\\
Term referring to a person who identifies their gender to be in line with the sex assigned to them at birth. For example, someone who was assigned female at birth and identifies as a woman is considered to be cisgender.\vspace*{\baselineskip}

\noindent\textbf{Gay}\\
Term referring to a homosexual person, frequently but not exclusively a homosexual man.\vspace*{\baselineskip}

\noindent\textbf{Gender}\\
Roles and identities that are socially constructed for men and women. Gender is not an inherent trait, but rather a fluid identity on the spectrum of socially defined femininity and masculinity.\vspace*{\baselineskip}

\noindent\textbf{Gender Identity}\\
An individual's feeling towards and experience of their personal gender. Gender identity does not have to be in line with the sex assigned at birth. For example, a person assigned female at birth can have a male gender identity.\vspace*{\baselineskip}

\noindent\textbf{Gender Minority}\\
An individual in a situation where their gender is not as widely represented as others. For example, a woman in physics is a gender minority.\vspace*{\baselineskip}

\noindent\textbf{Heterosexuality}\\
Sexual orientation characterized by attraction to members of another gender.\vspace*{\baselineskip}

\noindent\textbf{Homosexuality}\\
Sexual orientation characterized by attraction to members of the same gender.\vspace*{\baselineskip}

\noindent\textbf{Intersex}\\
A sex assigned at birth for persons exhibiting characteristics of both birth-assigned males and females, usually due to variations in prenatal development.\vspace*{\baselineskip}

\noindent\textbf{Lesbian}\\
Term referring to a homosexual woman.\vspace*{\baselineskip}

\noindent\textbf{LGBT+}\\
Lesbian, Gay, Bisexual, Transgender. The plus recognizes that not everyone fits their personal identity neatly into the LGBT constructs and may identify differently.\vspace*{\baselineskip}

%\noindent\textbf{oSTEM}\\
%Out in Science, Technology, Engineering, and Mathematics is a national society dedicated to the organization and professional development of LGBT students in STEM. The group consists of affiliate chapters throughout the U.S. and is led by a governing board known as oSTEM Incorporated.\vspace*{\baselineskip}

\noindent\textbf{Out (of the Closet)}\\
Openly identifying oneself as LGBT+. Someone may be out to some people but not to others (e.g. at school but not to family members, or vice versa). The decision to come out is highly personal. No one should be outed without their explicit prior agreement, as this can be harmful and even dangerous.\vspace*{\baselineskip}

\noindent\textbf{Transgender or Trans}\\
Term referring to a person who identifies their gender with one not in line with the sex assigned to them at birth. For example, a trans woman is someone who was assigned male at birth but whose gender identity is female.\vspace*{\baselineskip}

%\noindent\textbf{WISE}\\
%Women in Science and Engineering, a campus organization.\vspace*{\baselineskip}

\noindent\textbf{Sex}\\
A category, such as male, female, or intersex, assigned at birth based on biological characteristics.\vspace*{\baselineskip}

\noindent\textbf{Sexual Minority}\\
A person in a situation where their sexual orientation is not as widely represented as others. For example, a gay male in physics is a sexual minority.
\end{titlepage}








