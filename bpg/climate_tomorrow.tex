% vvvvvvvvvvvvvvvvvvvvvvvvvvvvvvvvvvvvvvvvvvvvvvvvvvvvvvvvvvvvvvvvvvvvvvvvvvvvvvvv
% Improving Departmental Climate Tomorrow (climate_tomorrow.tex)
%
% Chapter 2 of LGBTPhys-Org Best Practices Guide
% ^^^^^^^^^^^^^^^^^^^^^^^^^^^^^^^^^^^^^^^^^^^^^^^^^^^^^^^^^^^^^^^^^^^^^^^^^^^^^^^^

\chapter{Improving Departmental Climate Tomorrow}	% Chapter Title
\label{climate-tomorrow}		% Chapter Label
\normalsize			% Return to Normal font size


\section {Increase LGBT+ visibility within the department}
\label{visibility}
% vvvvvvvvvvvvvvvvvvvvvvvvvvvvvvvvvvvvvvvvvvvvvvvvvvvvvvvvvvvvvvvvvvvvvvvvvvvvvvv
Visibility and awareness are important aspects of promoting a positive departmental climate for LGBT+ people. Visibility and awareness of LGBT+ policies and even department members fosters an atmosphere of inclusion. This is particularly useful for students and faculty who worry about disclosure of their identities within the department. Visibility and awareness set the department's tone to be one of acceptance, encouragement, and focus on intellectual growth, regardless of identity and biology. There are many easy ways to increase visibility and awareness within departments.

The most important step is distributing university anti-discrimination policies early, often, and widely. This can be done through postings in faculty lounges and in student common areas, as well as notes in graduate student and faculty offer letters, and other correspondence. In their course syllabi, instructors can include both information on academic integrity and links to the non-discrimination policies of the university. A visiting weekend for graduate students is a great time to have a representative talk about diversity issues within the department and the larger community while also addressing the department's commitment to inclusion. Faculty candidate interviews are an appropriate time to include handouts about inclusionary policies such as health care policies, same-sex partner hiring, and gender-neutral bathroom options.

If your university does not have official inclusionary policies, the department can draft its own statement to explain its stance and approach to diversity and inclusion. This will help to attract the best candidates by showing a strong and supportive community.
% ^^^^^^^^^^^^^^^^^^^^^^^^^^^^^^^^^^^^^^^^^^^^^^^^^^^^^^^^^^^^^^^^^^^^^^^^^^^^^^^


\section {Create safe spaces within the department}
\label{safe-spaces}
% vvvvvvvvvvvvvvvvvvvvvvvvvvvvvvvvvvvvvvvvvvvvvvvvvvvvvvvvvvvvvvvvvvvvvvvvvvvvvvv
Navigating campus life poses at least some difficulties for most students, but is often even more fraught for those who also identify as members of sexual and gender minorities. Having a safe space for support and advice can make a big difference; just having such spaces available sends a powerful message of welcome and inclusion. On the other side, many well-meaning heterosexual, cisgender faculty and staff members are interested in assisting LGBT+ students and colleagues, but worry about saying or doing the wrong thing.

To address these concerns, many colleges and universities operate \textbf{Safe Zone} programs. These vary from institution to institution, but participants typically receive diversity training, a briefing on university resources available for LGBT+ students, and a sticker with which to prominently mark their offices as safe areas for people wishing to discuss LGBT+ issues. If a Safe Zone program is available on your campus, encourage faculty and staff -- especially those with administrative responsibilities, such as the department chair and student liaisons -- to undergo training and work to make their offices safe spaces.
% ^^^^^^^^^^^^^^^^^^^^^^^^^^^^^^^^^^^^^^^^^^^^^^^^^^^^^^^^^^^^^^^^^^^^^^^^^^^^^^^


\section {Increase networking opportunities}
\label{networking}
% vvvvvvvvvvvvvvvvvvvvvvvvvvvvvvvvvvvvvvvvvvvvvvvvvvvvvvvvvvvvvvvvvvvvvvvvvvvvvvv
The importance of forming effective networks for minorities and women in physics and astronomy is well understood. Scientific networks provide access to mentoring, job opportunities, material and emotional resources to support people faced with a challenging circumstance, potential collaborators and also recognition and dissemination of work. They can also be catalysts for instituting beneficial changes in policy. These same benefits hold true for LGBT+ people. While many institutions have valuable assets such as a Gay-Straight Alliance or an LGBT center, these groups are rarely able to support an individual simultaneously as a sexual/gender minority and also as a scientist. Moreover, these organizations do not always cater to graduate students, postdocs and faculty. There's a need, therefore, for networks that explicitly address all aspects of identity at an appropriate level for a person's career stage.

Supportive heads of department should reach out to the Gay-Straight Alliance or LGBT Center, if these exist at their institutions, to identify resources and networks that may already exist, and should suggest to leaders of those groups that it's necessary to make LGBT+ people feel welcome and supported explicitly in their scientific context. Students and faculty should be made aware of national networking organizations such as Out in Science, Technology, Engineering, and Mathematics (oSTEM) and the National Organization of Gay and Lesbian Scientists and Technical Professionals (NOGLSTP). Provide travel support for LGBT+ students to attend relevant networking meetings such as OUT for Work, NOGLSTP's OUT to Innovate conference, and oSTEM's national meeting, as is already common for networking conferences focused on scientists who are female, African-American, and/or Hispanic. Finally, if your school lacks an oSTEM chapter, encourage the formation of such a group.
% ^^^^^^^^^^^^^^^^^^^^^^^^^^^^^^^^^^^^^^^^^^^^^^^^^^^^^^^^^^^^^^^^^^^^^^^^^^^^^^^


\section {Help department members find resources}
\label{find-resources}
% vvvvvvvvvvvvvvvvvvvvvvvvvvvvvvvvvvvvvvvvvvvvvvvvvvvvvvvvvvvvvvvvvvvvvvvvvvvvvvv
As a chair, one of your primary roles is to help faculty and students in your department obtain local information and resources they need to be effective. This is especially important in supporting individuals belonging to populations (including LGBT+) that are traditionally under-represented in physics and astronomy. Research shows that these scientists are less likely to be part of the informal information-sharing networks through which those in the majority gain much of their information about how to survive and thrive in the profession\footnote{e.g., U. Sandstr\"{o}m, M. H\"{a}llsten. 2008. \emph{Persistent nepotism in peer-review}. Scientometrics 74 (2): 175-189.}.

Your mission, then, is to \textbf{learn what resources are available} on your campus and then to publicize them in a way that helps other faculty become part of the effort to be inclusive. As a starting point, consult the website of your local LGBT+ Resource Center (if one exists) or of the campus Diversity Office. Arrange to meet with the director of the center or office to learn more about how your campus is working to support LGBT+ faculty and students and how your department can join these efforts. Then bring this information back into your department by inviting the director to make a brief presentation at a faculty meeting or meet with interested student groups such as oSTEM or WISE. Publicize campus resources that would be of use to your students and faculty by referring to them in a prominent section of the departmental website or graduate student handbook - one visible to prospective as well as current department members.
% ^^^^^^^^^^^^^^^^^^^^^^^^^^^^^^^^^^^^^^^^^^^^^^^^^^^^^^^^^^^^^^^^^^^^^^^^^^^^^^^


\section {Encourage faculty and staff to receive diversity training}
\label{diversity-training}
% vvvvvvvvvvvvvvvvvvvvvvvvvvvvvvvvvvvvvvvvvvvvvvvvvvvvvvvvvvvvvvvvvvvvvvvvvvvvvvv
In seeking to develop an inclusive and supportive climate for LGBT+ members of one's department, it may be helpful to seek the assistance of a diversity training professional. This individual may provide specific sensitivity training for members of the department or offer other helpful resources. While members of one's department may have the best intentions regarding inclusivity, some may not be fully aware of unconscious assumptions or biases that, when inadvertently expressed, can contribute to an adverse or exclusionary climate. A diversity training session or workshop can help alert department members to such potentially unconscious biases and signals, provide a forum for educating department members about best practices, and offer an opportunity for discussion regarding LGBT+ inclusivity.

In some universities, there exists professional diversity training expertise on campus. Alternatively, a LGBT+ community center in the local community may provide contacts. These centers can be located through Center Link (see Appendix \ref{resources}). Webinars are also available through Campus Pride.
% ^^^^^^^^^^^^^^^^^^^^^^^^^^^^^^^^^^^^^^^^^^^^^^^^^^^^^^^^^^^^^^^^^^^^^^^^^^^^^^^


\section {Appoint a diversity liaison or committee}
\label{liaison}
% vvvvvvvvvvvvvvvvvvvvvvvvvvvvvvvvvvvvvvvvvvvvvvvvvvvvvvvvvvvvvvvvvvvvvvvvvvvvvvv
It can be valuable for a department to appoint a faculty member, or a small committee of faculty members, as a climate/diversity liaison, to be a confidential advisor and listener for faculty and students who may be having inclusionary issues within the department. These liaisons could be listed alongside policy postings and in syllabi, and be introduced at student gatherings and welcome events. The liaison should receive training for the role, e.g. through the local LGBT+ Center or Women's Resource Center, to ensure that they know how to be effective, how to maintain confidentiality, and how to steer people to appropriate campus or community resources. 

Such a diversity liaison needs to be seen as available and approachable for department members. They should initially introduce themselves to faculty and students and let people know how to contact them, and also renew these conversations over time so people remain mindful of their role. They should also send out regular communications (e.g., via e-mail or department newsletter) that emphasize the department's ongoing commitment to inclusion and share useful campus resources. Moreover, they should be proactive in seeking input on diversity issues from the faculty and students, and in communicating general trends or concerns to the chair to ensure that these issues receive timely attention. The diversity liaison should keep conversations confidential, except where the law or university policy require disclosure; when speaking with a student, staff member, or faculty member, the liaison should always make the limits of their confidentiality promise clear.
% ^^^^^^^^^^^^^^^^^^^^^^^^^^^^^^^^^^^^^^^^^^^^^^^^^^^^^^^^^^^^^^^^^^^^^^^^^^^^^^^


\section {Recognize and award significant achievements}
\label{recognize-achievements}
% vvvvvvvvvvvvvvvvvvvvvvvvvvvvvvvvvvvvvvvvvvvvvvvvvvvvvvvvvvvvvvvvvvvvvvvvvvvvvvv
Recognizing significant achievements of LGBT+ department members communicates that their contributions are valued equally along with those of others. Such recognition might include mention in a departmental newsletter or on a college or institutional website, nomination for a university or external prize, or an invitation to present a departmental colloquium.

The key point here is that one needs to make sure LGBT+ department members are fairly considered for such recognitions, alongside all other department members. For example, one might ask the departmental salary review committee to suggest nominees for various recognitions after reading everyone's files each year. Another possibility is to seek nominees from among current or recent candidates for reappointment, promotion, or tenure; each tenure-system faculty member will therefore be considered several times during their career. In the case of an award aimed at graduate teaching assistants, the graduate chair might look over the teaching evaluations of all TAs or contact supervisors of all more experienced TAs to get suggestions.

When considering a faculty member's service portfolio, work towards improving diversity or making the department climate more inclusive -- including for LGBT+ students, staff and faculty -- should be counted in the same way as any other service to the professional community.
% ^^^^^^^^^^^^^^^^^^^^^^^^^^^^^^^^^^^^^^^^^^^^^^^^^^^^^^^^^^^^^^^^^^^^^^^^^^^^^^^


\section {Include LGBT+ people in positions of power}
\label{positions-of-power}
% vvvvvvvvvvvvvvvvvvvvvvvvvvvvvvvvvvvvvvvvvvvvvvvvvvvvvvvvvvvvvvvvvvvvvvvvvvvvvvv
As with other under-represented groups in physics and astronomy, LGBT+ scientists may encounter barriers to their academic or career advancement by virtue of exclusion from positions of power or opportunities for recognition - a phenomenon known as the ``lavender ceiling". While an atmosphere of ``tolerance" or friendliness may exist on an individual or interpersonal level, full inclusivity can only occur when LGBT+ persons have \textbf{equal representation in structures} that provide access to power, resources, and recognition. Indeed, it has been documented that the experience or observation of exclusionary behavior within a department is significantly correlated with an LGBT+ faculty member's likelihood to leave their institution for an appointment elsewhere\footnote{R.S. Barthelemy, E.V. Patridge, S.R. Rankin, \emph{The Experience and Persistence of LGB STEM Faculty}. (article in preparation)}.

Thus, an important ``best practice" closely related to visibility is the inclusion of openly LGBT+ members of a department in positions of authority and power. Such positions include department chair, assistant chair, or chairs of key committees that affect departmental governance (e.g., hiring, strategic planning, graduate admissions) as well as other ad hoc roles that could enable LGBT+ persons to have equal voices within the department.
% ^^^^^^^^^^^^^^^^^^^^^^^^^^^^^^^^^^^^^^^^^^^^^^^^^^^^^^^^^^^^^^^^^^^^^^^^^^^^^^^

%\vspace{\baselineskip}
\section {Actively recruit LGBT+ students}
\label{recruit-students}
% vvvvvvvvvvvvvvvvvvvvvvvvvvvvvvvvvvvvvvvvvvvvvvvvvvvvvvvvvvvvvvvvvvvvvvvvvvvvvvv
Departments should actively recruit LGBT+ students. Such efforts not only build a more diverse body of students within the department by including those who are traditionally under-represented, but also increase the number of undergraduate majors by inviting students who might not otherwise think of physics and astronomy as hospitable careers. How to begin?  Send fliers to the LGBT+ center on campus, add a line to those flyers specifically welcoming LGBT+ students, and make departmental representatives visible at LGBT+ student events.  Invite willing and out students, faculty and staff to take on mentoring roles in the department.  Get in touch with (or become) the advisor of the campus oSTEM chapter.  Send a representative to recruit students at conferences like Out to Innovate or the national oSTEM meeting.  To broaden your department's reach, include information on inclusiveness and resources available to LGBT+ students in materials provided to prospective graduate students. 
% ^^^^^^^^^^^^^^^^^^^^^^^^^^^^^^^^^^^^^^^^^^^^^^^^^^^^^^^^^^^^^^^^^^^^^^^^^^^^^^^

\section {Allow name changes on departmental records}
\label{name-changes}
% vvvvvvvvvvvvvvvvvvvvvvvvvvvvvvvvvvvvvvvvvvvvvvvvvvvvvvvvvvvvvvvvvvvvvvvvvvvvvvv
Students, faculty members, and staff members sometimes change their names from those originally given at enrollment or hiring, for reasons including gender transition and marriage. This may not entail a legal name change for various reasons, including (but not limited to) concerns about family disclosure. Ensuring that an up-to-date, preferred name\footnote{See \href{http://www.itcs.umich.edu/itcsdocs/r1461/}{http://www.itcs.umich.edu/itcsdocs/r1461/} for an example of a preferred name policy.} is used for departmental records -- including directories, awards, office nameplates, and letters of reference -- is an especially vital practical concern for transgender department members, who may face discrimination in applications for employment or for further education. Establish a simple way for individuals to change their names in departmental files, and stress to faculty members that they should confidentially check with the student to determine which name and pronoun to use in reference letters. Always check with an individual before changing a name on any record, especially those that are publicly accessible.
% ^^^^^^^^^^^^^^^^^^^^^^^^^^^^^^^^^^^^^^^^^^^^^^^^^^^^^^^^^^^^^^^^^^^^^^^^^^^^^^^


\section {Consider LGBT+ persons when developing family-friendly policies}
\label{families}
% vvvvvvvvvvvvvvvvvvvvvvvvvvvvvvvvvvvvvvvvvvvvvvvvvvvvvvvvvvvvvvvvvvvvvvvvvvvvvvv
More and more of us are juggling child and/or elder care with our professional responsibilities.  In response, many employers are developing family-friendly policies to help employees balance their work and personal lives.  When developing such policies, be mindful of non-traditional families.  For example, explicitly include adoption in parental-leave policies, domestic partners in family-leave policies, and LGBT+ couples in dual-career accommodation practices.    Parental leave policies for many institutions are listed on the AstroBetter web site (\href{http://www.astrobetter.com/wiki/tiki-index.php?page=Leave+Policies}{astrobetter.com}).
% ^^^^^^^^^^^^^^^^^^^^^^^^^^^^^^^^^^^^^^^^^^^^^^^^^^^^^^^^^^^^^^^^^^^^^^^^^^^^^^^


\section {Increase protections for postdocs}
\label{post-docs}
% vvvvvvvvvvvvvvvvvvvvvvvvvvvvvvvvvvvvvvvvvvvvvvvvvvvvvvvvvvvvvvvvvvvvvvvvvvvvvvv

By its very nature, a postdoctoral fellowship is a vulnerable position.  Postdocs typically work at the pleasure of their advisor and are paid out of his or her grant.  Advisors may feel a tension between their obligation to advance the careers of their postdocs and their desire to compete a particular project before the money runs out.  To improve the oversight and advancement of its postdocs, the Space Telescope Science Institute recently began a postdoctoral mentorship program, assigning to each postdoc a mentor who is \emph{not} his or her advisor.  The program has sponsored workshops on topics ranging from ``How to Find a Job'' to ``How to Negotiate a Salary''.  Such mentoring programs can be replicated at your institution.

Postdocs may not receive the same job protections offered to other university employees.  For example, while the Family and Medical Leave Act (FMLA) provides certain employees with up to 12 weeks of unpaid, job-protected leave per year, postdocs are often not considered employees and thus are not necessarily guaranteed FMLA protections.  Paid medical leave would be extremely helpful for postdocs taking time off to recover from gender-alignment surgery or to care for a new child.


\section {Keep people safe when traveling}
\label{accommodations}
% vvvvvvvvvvvvvvvvvvvvvvvvvvvvvvvvvvvvvvvvvvvvvvvvvvvvvvvvvvvvvvvvvvvvvvvvvvvvvvv
Many research groups expect group members of the same (perceived) sex to bunk together at conferences to reduce travel expenses.  What if one's perceived sex differs from one's gender identity?  This is an even bigger issue when people are expected to be away from their home institution for a long time, as at a remote observing site or an accelerator. When the group rents an apartment, what rooming situation do they expect people to live in? How does the group deal with harassment complaints if there are two grad students on site for two months and no senior members of the group?  A departmental policy regarding travel accommodations can prevent embarrassment, trauma -- and possible litigation.
% ^^^^^^^^^^^^^^^^^^^^^^^^^^^^^^^^^^^^^^^^^^^^^^^^^^^^^^^^^^^^^^^^^^^^^^^^^^^^^^^















