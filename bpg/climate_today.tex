% vvvvvvvvvvvvvvvvvvvvvvvvvvvvvvvvvvvvvvvvvvvvvvvvvvvvvvvvvvvvvvvvvvvvvvvvvvvvvvvv
% Improving Departmental Climate Today (climate_today.tex)
%
% Chapter 1 of LGBTPhys-Org Best Practices Guide
% ^^^^^^^^^^^^^^^^^^^^^^^^^^^^^^^^^^^^^^^^^^^^^^^^^^^^^^^^^^^^^^^^^^^^^^^^^^^^^^^^

\chapter{Improving Departmental Climate Today}	% Chapter Title
\label{climate-today}		% Chapter Label
\normalsize			% Return to Normal font size



\section {Include everyone in social events}
\label{social-events}
% vvvvvvvvvvvvvvvvvvvvvvvvvvvvvvvvvvvvvvvvvvvvvvvvvvvvvvvvvvvvvvvvvvvvvvvvvvvvvvv
Department social events, whether on or off campus, are important opportunities for faculty, staff and students not only to network, but also to form a real community. Ensure that LGBT+ department members and their spouses, partners, and children are explicitly and implicitly invited to and welcome at these events, in the same way as their heterosexual and cisgender peers. For example, instead of inviting ``spouses", \textbf{invite ``spouses and partners" or ``significant others"}. This practice is especially important for new department members and for newly out department members. 
% ^^^^^^^^^^^^^^^^^^^^^^^^^^^^^^^^^^^^^^^^^^^^^^^^^^^^^^^^^^^^^^^^^^^^^^^^^^^^^^^

\section {Use gender-neutral and inclusive language}
\label{gender-language}
% vvvvvvvvvvvvvvvvvvvvvvvvvvvvvvvvvvvvvvvvvvvvvvvvvvvvvvvvvvvvvvvvvvvvvvvvvvvvvvv
While it is true that most people in our society are heterosexual and cisgender, not everyone is. The heterosexual and cisgender norm is often unwittingly reinforced through our use of language. This can leave LGBT+ people feeling excluded. Here are some suggestions for gender-neutral and inclusive language:

\begin{itemize}
	\item Remember that there is a difference between a person's gender (culturally determined) and a person's sex (biologically defined). Gender is not a binary, but rather a continuum.
	\item \textbf{Use gender-neutral pronouns and phrasing} such as ``Bring your partner" instead of ``Bring your wife", or ``All students should bring their laptops" instead of ``Each student should bring his laptop").
	\item \textbf{Always use the name and pronoun of a person's choosing}. If you are unsure which pronoun a person prefers, try to avoid using one until you can ask the person in private, ``How would you prefer to be addressed?" At the beginning of a semester, distribute a form to all students which asks for their preferred name and pronouns along with any other information that an instructor might need (such as whether the student is on a sports team).
	\item \textbf{Avoid terms that sustain gender biases when describing titles or professions}. For example, use ``chair" instead of ``chairman", and ``custodian" instead of ``cleaning lady".
	\item Avoid defaulting to umbrella terms like ``gay" or ``homosexual." Use ``LGBT" to refer to a broad community.
	\item \textbf{Do not assume all people have a heterosexual orientation}.  Do not even make that assumption about everyone in the room.
	\item Remember that the term ``sexual orientation" is preferred over ``sexual preference"; the latter suggests a degree of voluntary choice that is not necessarily the case.
\end{itemize}
% ^^^^^^^^^^^^^^^^^^^^^^^^^^^^^^^^^^^^^^^^^^^^^^^^^^^^^^^^^^^^^^^^^^^^^^^^^^^^^^^


\section {Do not tolerate offensive language}
\label{offensive-language}
% vvvvvvvvvvvvvvvvvvvvvvvvvvvvvvvvvvvvvvvvvvvvvvvvvvvvvvvvvvvvvvvvvvvvvvvvvvvvvvv
As a department, adopt a policy that racist, sexist, homophobic, and ethnic slurs and jokes are unprofessional and will not be tolerated.  Base the policy on the institution's non-discrimination statement and on the need for a departmental climate in which all are welcome and encouraged to do great science.  Once a department embraces such a policy, peer pressure can do most of the enforcement.  Remind department members about this policy regularly.  Make sure that incoming folks, especially graduate students, are aware.

In personal interactions, point out offensive language and ask that it stop.  Make clear that such language is unprofessional and unwelcome in your department.
% ^^^^^^^^^^^^^^^^^^^^^^^^^^^^^^^^^^^^^^^^^^^^^^^^^^^^^^^^^^^^^^^^^^^^^^^^^^^^^^^


\section {Invite LGBT+ speakers to campus}
\label{lgbt-speakers}
% vvvvvvvvvvvvvvvvvvvvvvvvvvvvvvvvvvvvvvvvvvvvvvvvvvvvvvvvvvvvvvvvvvvvvvvvvvvvvvv
One way to help those belonging to marginalized populations become more integrated into the academic community is to recognize them publicly for their professional accomplishments. This recognition provides other members of the community with role models with whom they can identify. The APS has, for years, publicized speakers' lists of women and minority physicists in order to encourage departments to diversify their colloquium and seminar series.  Similarly, when a department invites a speaker from the LGBT+ community to make a research presentation, it simultaneously showcases that individual's work, provides them with networking opportunities in the department, and offers local students (and even faculty) a role model. It also enables the department to demonstrate publicly its commitment to inclusive excellence.

When \textbf{inviting a speaker to campus}, it is always good practice to arrange for them to meet with individuals or groups with whom they have common interests. For example, an invited speaker who is an expert in instructional methods may wish to meet with fellow educators. If you are hosting someone whom you know to be a public advocate for LGBT+ concerns, take all dimensions of their portfolio into account in constructing their schedule but keep in mind that that outing a person can have serious consequences. With the speaker's permission, \textbf{provide a mini-bio that references their interest in LGBT+ issues} as well as their scientific accomplishments; this will encourage a wider range of individuals to come to their talk or seek to meet with them. Ask the visitor if they are willing to meet with any interested student or faculty groups -- sharing pizza and conversation with the condensed-matter graduate students, a local chapter of oSTEM, or the campus Women in Science and Engineering (WISE) group may be a valuable experience for all concerned.
% ^^^^^^^^^^^^^^^^^^^^^^^^^^^^^^^^^^^^^^^^^^^^^^^^^^^^^^^^^^^^^^^^^^^^^^^^^^^^^^^

\section {Pay attention to course climate}
\label{course-climate}
% vvvvvvvvvvvvvvvvvvvvvvvvvvvvvvvvvvvvvvvvvvvvvvvvvvvvvvvvvvvvvvvvvvvvvvvvvvvvvvv
Course climate refers to how welcoming a course as a whole is to students of all backgrounds and identities\footnote{A highly recommended review may be found in Chapter 6 of S.A. Ambrose, et al. 2010. \emph{Why do student development and course climate matter for student learning}. In \textbf{How Learning Works}. 1st ed. Jossey-Bass.}: it is influenced by a number of factors, including choice of subject matter, attitude and language used by the instructor and TAs, as well as the nature of interactions among students. Each student may perceive the climate differently, experiencing anything from overt hostility or discrimination to implicit marginalization to an explicitly welcoming environment\footnote{C. DeSurra, K.A. Church. 1994. \emph{Unlocking the classroom closet: Privileging the marginalized voices of gay/lesbian college students}. Paper presented to the Annual Meeting of the Speech Communication Association.}. A student who identifies as a minority (interpreted broadly to include race, gender, class, LGBT+ status, religion, nationality) is particularly likely to experience a negative climate due to the use of stereotypes\footnote{C.M. Steele, J.R. Aronson. 1995. \emph{Stereotype threat and the intellectual performance of African Americans}. Journal of Personality and Social Psychology 69 (5): 797-811.} and prior assumptions on the part of the instructor(s) about the students in the classroom. For instance, the student may overhear classmates using sexist, racist, or homophobic language, be the direct target of such remarks, or feel excluded by classmates during team projects or group work. Over time, these experiences can have a corrosive effect.   In contrast, an instructor who promotes professional behavior in the classroom regularly calls on all students to ask or answer questions, creates opportunities for each student to discuss their work with members of the course staff, and encourages all team members to work together to create a climate where every student feels intellectually valued.

Minority-identifying students may face additional hurdles to successful learning in a negative climate because their emotional reactions can disrupt their cognitive processes. If the classroom climate is hostile, they will be less likely to ask questions, join study groups, or attend faculty office hours and more likely to skip class sessions altogether; these patterns can lead students to lag behind and underachieve in the course. This can be exacerbated if the student has been experiencing rejection outside the classroom as well (e.g., lack of support from family or friends). Ultimately, affected students may lose their motivation to continue with their chosen discipline and switch to one where they perceive the climate to be more congenial\footnote{B. Major, S. Spencer, T. Schmader, C. Wolfe, J. Crocker. 1998. \emph{Coping with negative stereotypes about intellectual performance: The role of psychological disengagement}. Personality and Social Psychology Bulletin 24 (1): 34-50}. Particular challenges exist for improving climate for LGBT+ students because they are less likely to be visible than other minorities, discrimination against LGBT+ individuals is still pervasive, and relatively few role models of LGBT+ scientists are presently available.

To identify possible issues, provide anonymous feedback mechanisms for students to report climatic issues arising in the classroom. These should be in place and announced early in each term. It is helpful to have a standard slide for use on the first day of all classes in the department, identifying classroom standards and contact information for feedback or complaints. A departmental diversity liaison may facilitate this. If you believe there is a broader problem, consider having your institution's Diversity Office, Women's Resource Center, or LGBT+ Resource Center conduct a climate survey.
% ^^^^^^^^^^^^^^^^^^^^^^^^^^^^^^^^^^^^^^^^^^^^^^^^^^^^^^^^^^^^^^^^^^^^^^^^^^^^^^^

\section {Discuss climate with faculty and TAs}
\label{train-faculty}
% vvvvvvvvvvvvvvvvvvvvvvvvvvvvvvvvvvvvvvvvvvvvvvvvvvvvvvvvvvvvvvvvvvvvvvvvvvvvvvv
The department should encourage faculty and TAs to educate themselves about the impact of course climate on minority students, and to make their classrooms more welcoming for them. Possible ways to raise awareness of this include: discussing the issue at a department meeting or teaching seminar, including a suitable book or article into journal clubs, or inviting an education researcher to give a colloquium. Some topics to discuss are the language used in the classroom, breadth of role models available to students, inclusion of welcoming language in course syllabi, and prior assumptions implicit in questions.

Instructors can create an inclusive learning environment within individual courses using a variety of research-based techniques. Some of these, such as interactive pedagogical methods, both increase the degree to which all students learn and also have been found to alleviate gender gaps in student performance in introductory physics classes\footnote{M. Lorenzo, et al. \emph{Reducing the gender gap in the physics classroom}. American Journal of Physics 74 (2) 118}. Others are more specifically aimed at countering stereotype threats and other barriers to the success of students from under-represented groups. University Teaching Centers or LGBT+ Resource Centers may be able to provide suitable training sessions (e.g. for Safe Zone programs) or even fellowship programs to help instructors learn these techniques. 

Encourage faculty to share examples with their colleagues and TAs of how they use these research-based methods to integrate positive diverse role models of physicists and astronomers into their classes. This can help build a more inclusive teaching culture in the department as a whole.
% ^^^^^^^^^^^^^^^^^^^^^^^^^^^^^^^^^^^^^^^^^^^^^^^^^^^^^^^^^^^^^^^^^^^^^^^^^^^^^^^

%\section {Seek feedback on classroom climate}
%\label{climate-feedback}
% vvvvvvvvvvvvvvvvvvvvvvvvvvvvvvvvvvvvvvvvvvvvvvvvvvvvvvvvvvvvvvvvvvvvvvvvvvvvvvv
% ^^^^^^^^^^^^^^^^^^^^^^^^^^^^^^^^^^^^^^^^^^^^^^^^^^^^^^^^^^^^^^^^^^^^^^^^^^^^^^^

\section {Discuss climate with advisees}
\label{talk-advisees}
% vvvvvvvvvvvvvvvvvvvvvvvvvvvvvvvvvvvvvvvvvvvvvvvvvvvvvvvvvvvvvvvvvvvvvvvvvvvvvvv
Students and junior scientists who face a hostile climate -- in the classroom, in the laboratory, or in the department as a whole -- are often hesitant to raise the problem with an authority figure. They may worry that the climate is somehow their fault; that even terrible harassment is not bad enough to justify involving a busy, more powerful person; or that there is no possibility for positive change. They may feel pressure, external or internal, not to get a labmate or classmate into trouble, or they may believe that it is up to them to manage any harassment they face on their own. This type of problem almost never goes away by itself, however, and an instructor, supervisor, or department chair, or other identified liaison is a natural person to intervene in an unhealthy climate.

Encourage instructors, advisors, supervisors, and mentors to ask about climate in their regular meetings with students, trainees, or other junior scientists. Helpful questions include: Do they feel welcomed in the department or lab group? Is the climate at homework-help sessions conducive to their learning? Have they or others experienced harassment or belittlement by peers or by more junior or senior individuals? What kind of changes are needed? It is important to ask questions like these of everyone, since people may be targeted on many different grounds -- sexual orientation, gender identity/expression, race, ethnicity, nationality, sex, disability, religion, etc. By expressing a sincere interest in junior scientists' experiences, mentors can indicate that they will listen sympathetically and are willing to take active steps to solve any problems that arise. Complaints must be taken seriously and a complainant should be able to expect an improvement in their circumstances. A harasser may or may not intend to harm, but that does not make their actions less damaging. No one should feel that they must hide their identity in order to study or work in physics or astronomy.

Ensure that there are also avenues for complaints outside the normal advising hierarchy. Undergraduate- and graduate-student liaisons are natural candidates for this role, as are university trainers in diversity and ethics. Many departments have annual or semi-annual orientations, social events, or mailers, which are good opportunities to remind students, faculty and staff of the expectations of the department, and the resources that are available to them.

% ^^^^^^^^^^^^^^^^^^^^^^^^^^^^^^^^^^^^^^^^^^^^^^^^^^^^^^^^^^^^^^^^^^^^^^^^^^^^^^^


\section {Understand departmental demographics}
\label{demographics}
% vvvvvvvvvvvvvvvvvvvvvvvvvvvvvvvvvvvvvvvvvvvvvvvvvvvvvvvvvvvvvvvvvvvvvvvvvvvvvvv
Consider how internal demographic information and/or demographic information from job applicants and prospective students may be collected in an inclusive way. Does the department's demographic form include a question about sexual orientation and a question about gender identity? Can respondents list a domestic partnership as a marital status? Are respondents limited to a binary, male or female identification, or can they write in how they self-identify?

Internal questionnaires may include such questions as
\begin{itemize}
	\item What is your gender? Male / Female / Non-binary / Other:\noindent\underline{\makebox[0.5in][l]{}}
	\item Do you consider yourself a member of the LGBT (lesbian, gay, bisexual, transgender) community? Yes / No
\end{itemize}

\noindent You may wish to include definitions of terms with the survey.

As with other diversity questions, responses to such queries provide valuable statistical information, but can also pose risks for the respondent. Survey responses should be anonymized; forms with these questions should clearly indicate what will happen to the data so that respondents can feel confident about how their answers will be used. All such data should be \textbf{separated from any decision-making} related to hiring, awards, or promotions. Demographic data should be passed to Human Resources or to a designated collator and kept separately from other materials. Do not collect this information if you cannot prevent its misuse.
% ^^^^^^^^^^^^^^^^^^^^^^^^^^^^^^^^^^^^^^^^^^^^^^^^^^^^^^^^^^^^^^^^^^^^^^^^^^^^^^^


\section {Participate in surveys exploring LGBT+ experiences}
\label{surveys}
% vvvvvvvvvvvvvvvvvvvvvvvvvvvvvvvvvvvvvvvvvvvvvvvvvvvvvvvvvvvvvvvvvvvvvvvvvvvvvvv
Data collection is a vital component of diversity efforts. For any individual department or organization, it is necessary to evaluate the effects of existing policies and identify areas where improvement is required. For the larger academic community, an extensive, reliable data set allows constructive comparisons among departments and institutions, which may guide policy-making or even career decisions. The inclusion of LGBT+ demographic information and LGBT+ experiences in data collection is thus an essential element of formulating policies that are friendly to LGBT+ students, faculty and staff.

An important aspect of being a supportive chair is helping with the dissemination of research surveys. No data currently exists on the numbers of LGBT+ physicists and astronomers and their experiences within the academy. When you receive an email message asking you and your department members (students, faculty and/or staff) to participate in a survey, it is important that you distribute the email widely. This will help the greater community to collect the data necessary to understand what is actually happening and what issues need to be addressed.
% ^^^^^^^^^^^^^^^^^^^^^^^^^^^^^^^^^^^^^^^^^^^^^^^^^^^^^^^^^^^^^^^^^^^^^^^^^^^^^^^


\section {Join an Out List as an LGBT+ scientist or as an ally}
\label{outlist}
% vvvvvvvvvvvvvvvvvvvvvvvvvvvvvvvvvvvvvvvvvvvvvvvvvvvvvvvvvvvvvvvvvvvvvvvvvvvvvvv
Finding a mentor who is knowledgeable about and can address the concerns of an LGBT+ student can be a difficult process. When the LGBT+ Physicists group was created in 2010, one of the major concerns raised by attendees and members was networking and finding other LGBT+ people in the field. Before these conversations began, most LGBT+ physicists had not met another LGBT+ person in the field during their career.  Similarly, when LGBT+ astronomers began meeting informally at scientific conferences in 1992, their principal motivation was simply to develop a community.

Another way to raise visibility and provide targeted mentorship is to place one's name on a public Out List either as an LGBT+ scientist or as an ally. These lists allow students to find mentors and show leadership from allies. Some institutions already have such lists in place. There are also national lists, such as the \href{http://lgbtphysicists.org/outlist.html}{LGBT+ Physicists Out List} and the \href{http://web.physics.ucsb.edu/~blaes/lgbtastro/}{Outlist of LGBT Astronomers}. Signatories to these lists have stated a commitment to working against bias and discrimination in the fields of physics and astronomy.

As a potential mentor and ally, familiarize yourself with societies and organizations that work on behalf of LGBT+ people, both on your campus and in general, so that you may recommend them to students, staff and faculty members who ask.
% ^^^^^^^^^^^^^^^^^^^^^^^^^^^^^^^^^^^^^^^^^^^^^^^^^^^^^^^^^^^^^^^^^^^^^^^^^^^^^^^

