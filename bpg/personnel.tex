% vvvvvvvvvvvvvvvvvvvvvvvvvvvvvvvvvvvvvvvvvvvvvvvvvvvvvvvvvvvvvvvvvvvvvvvvvvvvvvvv
% Personnel Issues (personnel.tex)
%
% Chapter 3 of LGBTPhys-Org Best Practices Guide
% ^^^^^^^^^^^^^^^^^^^^^^^^^^^^^^^^^^^^^^^^^^^^^^^^^^^^^^^^^^^^^^^^^^^^^^^^^^^^^^^^

\chapter{Personnel Issues}	% Chapter Title
\label{personnel-issues}		% Chapter Label
\normalsize			% Return to Normal font size

\section [* Include non-discrimination statements in job announcements]{* Include non-discrimination statements in job \\announcements}
\label{nondisc-statement}
% vvvvvvvvvvvvvvvvvvvvvvvvvvvvvvvvvvvvvvvvvvvvvvvvvvvvvvvvvvvvvvvvvvvvvvvvvvvvvvv
Including a brief statement on the EEO policy of the employer in the job announcement serves several goals. It clarifies the legal situation that a potential employee enters, but it also serves to signal potential employees that the employer is aware of the issues facing LGBT+ people. If competing institutions lack protections or partner benefits for LGBT+ people, qualified LGBT+ applicants may be attracted to a non-discriminating institution they might otherwise have overlooked.

Employers can include a brief EEO statement stating that ``this employer prohibits discrimination based on sexual orientation, gender identity, and gender expression." In individual job postings, employers can include language to point out that they ``encourage applications from eligible candidates regardless of gender, race, national origin, age, religion, marital status, political views, sexual orientation, gender identity, gender expression, or disability."
% ^^^^^^^^^^^^^^^^^^^^^^^^^^^^^^^^^^^^^^^^^^^^^^^^^^^^^^^^^^^^^^^^^^^^^^^^^^^^^^^

\newpage
\section {* Avoid assumptions}
\label{assumptions}
% vvvvvvvvvvvvvvvvvvvvvvvvvvvvvvvvvvvvvvvvvvvvvvvvvvvvvvvvvvvvvvvvvvvvvvvvvvvvvvv
It is always embarrassing to use the wrong title, name, nickname or pronoun to address or refer to someone. When the individual in question is LGBT+, this type of mistake can be particularly hurtful.

To avoid such errors, beware of assigning pronouns to people you have not met. In conversations and deliberations, consider each applicant by name until as late as possible in the process. This precaution also helps reduce potential gender bias in the hiring process. When you make direct contact with an applicant for the first time (e.g. in a telephone or in-person interview), ask ``How would you prefer to be addressed?" and then communicate this information to other department members involved in the hiring process. This simple question accommodates a wide range of situations, from gender expression to nicknames.
% ^^^^^^^^^^^^^^^^^^^^^^^^^^^^^^^^^^^^^^^^^^^^^^^^^^^^^^^^^^^^^^^^^^^^^^^^^^^^^^^


\section {* Be open to name changes for job and tenure applicants}
\label{job-name-change}
% vvvvvvvvvvvvvvvvvvvvvvvvvvvvvvvvvvvvvvvvvvvvvvvvvvvvvvvvvvvvvvvvvvvvvvvvvvvvvvv
Anyone who has changed names may encounter challenges when applying for employment, awards, or promotions. Employers are increasingly likely to require a background check as part of the application process; this entails providing all of the legal names one has held to the agency doing the background check. Evaluation committees typically require that one submit a list of publications as part of a job, award, or promotion application. It is conventional to provide the names of the authors so that the evaluators can note their relative seniority and/or ordering as part of assessing the candidate's relative contributions to the work. People change their names for many personal reasons, including witness protection, entering a life partnership (usually a change of last name) or undergoing a gender transition (usually a change of first name). Since these reasons are not generally relevant to job qualifications, and tend to reveal information that the employee may prefer to keep private, employers should minimize the instances in which employees must reveal the history of their names.

Departments can take steps to balance employees' reasonable privacy concerns against the requirements of employment and evaluation processes. In the case of background checks, the department should already be ensuring that only those directly involved in performing the check see any information that the individual submits. In the case of evaluations for awards or tenure, the department could explicitly establish a convention of including only the last (family) names of all authors on publication lists. This would still enable experts in the field to evaluate the author ordering and seniority of collaborators, while protecting transgender individuals from being forced to out themselves. Since this would still show where an individual has changed their family name, raising issues of gender bias, the department could issue a statement such as:
\begin{quote}
	\emph{In publication lists for award or tenure applications, please list all authors' last names only and put your own last name in bold type to make it easy for the readers to find. \textbf{Name changes are not relevant to our decision and will not be considered in the evaluation}. Please also include a brief statement at the start of the publication list that notes the author ordering convention in your sub-field (e.g., alphabetical, students first, primary author first, etc.).}
\end{quote}
This establishes that the department will only consider professionally relevant information and offers a practical way for individuals to handle several name-related issues that frequently arise.

Finally, it should be noted that addressing naming issues in one's CV or publication list does not cover all eventualities. Evaluators who look up a journal article may still discover that someone has changed their name. Individuals may wish to contact their publishers to investigate the possibility of updating their name on past publications.

% ^^^^^^^^^^^^^^^^^^^^^^^^^^^^^^^^^^^^^^^^^^^^^^^^^^^^^^^^^^^^^^^^^^^^^^^^^^^^^^^


\section {Provide help for all dual-career couples}
\label{dual-career}
% vvvvvvvvvvvvvvvvvvvvvvvvvvvvvvvvvvvvvvvvvvvvvvvvvvvvvvvvvvvvvvvvvvvvvvvvvvvvvvv
For any dual-career partners, decisions about employment opportunities can be affected by uncertainty about the career prospects for the partner who is not the one primarily being recruited. In the case of same-sex dual-career partners these problems can be amplified, especially in states or at institutions where the couple's relationship might not be recognized.  

Discussing dual career issues before an offer of employment has been made is challenging.  The potential employer is legally barred from inquiring about the personal life of the job candidate. Additionally, the candidate may not wish to raise these issues, lest they influence the likelihood of receiving an offer.  Therefore, the department chair should make it standard practice to inform all job candidates or finalists about general university resources related to work-life balance; the chair should state clearly that this one-way flow of information (from chair to candidate) is standard practice.  For instance, the chair might provide a copy of a university work-life resource guide, links to the local HERC\footnote{Higher Education Recruitment Consortium; \href{http://hercjobs.org}{http://hercjobs.org}} website, or the contact information of the university's point person for dual career issues. 

Note that even when a potential employee is comfortable discussing dual-career issues with potential employers (e.g., after a formal offer is in hand), a satisfactory solution may be impeded or precluded by legal barriers.
% ^^^^^^^^^^^^^^^^^^^^^^^^^^^^^^^^^^^^^^^^^^^^^^^^^^^^^^^^^^^^^^^^^^^^^^^^^^^^^^^

















