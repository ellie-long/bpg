%\frontmatter
\begin{titlepage}
%\thispagestyle{empty}
\setcounter{page}{3}
\begin{center}

{\Large{\textbf{Supporting LGBT+ \\Physicists \& Astronomers:\\ Best Practices for Academic Departments}}}
\vspace*{\baselineskip}

{\large{\textbf{Introduction}}}
\vspace*{\baselineskip}
\end{center}

When physicists and astronomers discuss issues related to diversity or broadening participation in the field, the focus is typically on creating support mechanisms for women or people of color.  However, scientists who identify as lesbian, gay, bisexual, or transgender (LGBT) are also a minority within the physics and astronomy communities and can find themselves marginalized in various ways.  This document aims to highlight opportunities for making our workplaces more inclusive for LGBT+ scientists.  (We add a plus symbol to remind ourselves that not everyone fits neatly into the LGBT constructs, and some may identify differently.)\vspace*{\baselineskip}

The argument for inclusion is simple: science advances fastest when the best scientists are free to apply their intelligence and imagination to the exploration of the universe without limits and without fear.  Sometimes, the best scientists are LGBT+.  Institutions that are viewed as unfriendly to LGBT+ people quickly find themselves at a competitive disadvantage.  When LGBT+ scientists leave our departments to work at other institutions, our students, our scholarly communities, and our own research suffer.  Furthermore, a more inclusive workplace has advantages for all of us: greater flexibility to perform our work, greater support for work/life issues, and greater freedom to be ourselves. \vspace*{\baselineskip}

Best practices for the inclusion of LGBT+ people on campus have been proposed by several authors\footnote{W.J. Blumenfeld. 1993. \emph{Making Colleges and Universities Safe for Gay and Lesbian Students: Report and Recommendations of the Massachusetts Governors Commission on Gay and Lesbian Youth}. Boston, Massachusetts.}$^,$\footnote{S. Rankin. 2003. \emph{Campus Climate for Gay, Lesbian, Bisexual, and Transgender People: A National Perspective}. New York: National Gay and Lesbian Task Force Policy Institute.}$^,$\footnote{S. Windmeyer, S. Rankin, G. Beemyn. 2009. \emph{Campus Climate Index}. \href{http://www.campusprideindex.org}{campusprideindex.org}.}. In this document, we limit ourselves to recommendations that are particularly relevant to faculty and department chairs (as opposed to university administrators). After a brief glossary of terms, we offer both short-term and long-term department-level suggestions, then address recruitment and personnel issues.  We conclude with recommendations for university-level policies that may guide conversations with institutional administrators. A list of useful external resources is available at the end of the document, along with the author list.\vspace*{\baselineskip}

\end{titlepage}