%\frontmatter
\begin{titlepage}
%\thispagestyle{empty}
\setcounter{page}{3}
\begin{center}

%{\Large{\textbf{Supporting LGBT+ \\Physicists \& Astronomers:\\ Best Practices for Academic Departments}}}
%\vspace*{\baselineskip}

{\large{\textbf{Introduction}}}
\vspace*{\baselineskip}
\end{center}

When physicists and astronomers discuss issues related to diversity or broadening participation in the field, the focus is typically on creating support mechanisms for women or people of color.  However, scientists who identify as lesbian, gay, bisexual, or transgender (LGBT) are also a minority within the physics and astronomy communities and can find themselves marginalized in a variety of ways.  This document aims to highlight opportunities for making academic departments more welcoming for LGBT+ students, staff, and faculty.  (In this document, we add a plus symbol to remind ourselves that not everyone fits neatly into the LGBT constructs, and some may identify differently.) \vspace*{\baselineskip}

Even if you consider your department to be a safe and welcoming space, it is important to note that the campus environment is often perceived differently by different groups: A 2003 study of campus climate\footnote{\link{http://www.thetaskforce.org/downloads/reports/reports/CampusClimate.pdf}{S. R. Rankin. 2003. \emph{Campus Climate for Gay, Lesbian, Bisexual, and Transgender People: A National Perspective}. New York: National Gay and Lesbian Task Force Policy Institute.}} found that, while 90\% of heterosexual students classified their campuses as friendly, 74\% of LGBT students rated the campus climate as homophobic. LGBT staff and faculty routinely report that their campuses are more homophobic than reported by students. A 2010 study\footnote{\link{http://issuu.com/campuspride/docs/campus_pride_2010_lgbt_report_summary/13?e=0}{{S.~R. Rankin, G. Weber, W. Blumenfeld, \& S. Frazer. 2010.} {\em State of Higher Education for Lesbian, Gay, Bisexual and Transgender People. Charlotte, NC: Campus Pride}.}} found that 23\% of LGB respondents had been harassed within the past year, with even higher rates (31--39\%) for transgender individuals. Around half of all LGB students and two-thirds of transgender students had avoided disclosing their identity to avoid harassment. \vspace*{\baselineskip}

The argument for inclusion is simple:  Science advances fastest when scientists are free to apply their intelligence and imagination to the exploration of the universe without limits and without fear.  Sometimes, those scientists are LGBT+.  Institutions that are viewed as unfriendly to LGBT+ people quickly find themselves at a competitive disadvantage.  When LGBT+ scientists leave our departments to work at other institutions, our students, our scholarly communities, and our own research suffer.  Furthermore, a more inclusive workplace has advantages for all of us: greater flexibility to perform our work, greater support for work/life issues, and greater freedom to be ourselves. \vspace*{\baselineskip}

Recent studies suggest that, over the next decade, the U.S.\ will need a million more college graduates in the science, technology, engineering, and mathematics (STEM) fields than previously expected.  This gap could largely be filled simply by increasing the retention rate of undergraduates majoring in STEM fields from 40 to 50\%.\footnote{\link{http://www.whitehouse.gov/sites/default/files/microsites/ostp/pcast-executive-report-final_2-13-12.pdf}{{President's Council of Advisor on Science and Technology}, \emph{Engage to Excel: Producing One Million Additional College Graduates with Degrees in Science, Technology, Engineering, and Mathematics}.}}  Past initiatives to increase the participation of underrepresented groups in STEM fields have ignored LGBT+ people.  Given our need to remain competitive in a global economy, it seems prudent to increase the recruitment and retention of STEM talent from all demographics\footnote{\link{http://www.dl.begellhouse.com/references/00551c876cc2f027,761a7b37493b2d86,6fe4cda94f55abdf.html}{{E.~V. Patridge, R.~S. Barthelemy, \& S.~R. Rankin. 2014.} Factors Impacting the Academic Climate for LGBQ STEM Faculty. {\em Journal of Women and Minorities in Science and Engineering}\/ 20: 75-98}.}. \vspace*{\baselineskip}

Research has shown that the presence of ``difference'' on campus is important to the intellectual and social development of the majority\footnote{R.~D. Reason, B.~E. Cox, B.~R.~L. Quaye, \& P.~T. Terenzini. 2010. Faculty and institutional factors that promote student encounters with difference in first-year courses. {\em The Review of Higher Education}, 33(3), 391-414.}$^,$\footnote{H. Smith, R. Parr, R. Woods, B. Bauer, \& T. Abraham. 2010. Five years after graduation: Undergraduate cross-group friendships and multicultural curriculum predict current attitudes and activities. {\em Journal of College Student Development}, 51(4), 385-402.}. Specifically, students who interact with people different from themselves show more developed critical and creative thinking, improved cross-cultural relationships, and increased volunteerism and civil activism. Students and staff who encounter contradictions in their own thinking around LGBT+ issues can develop greater levels of cognitive sophistication\footnote{P. Love. 1997. Contradiction and paradox: Attempting to change the culture of sexual orientation at a small Catholic college. {\em The Review of Higher Education}, 20(4), 381-398.}. The advantages of engaging with ``difference'' appear to continue beyond college and into the workforce\footnote{Smith et al.}. A diverse department can thus improve your students' performance, their ability to function in a multicultural workforce, and the reputation of your department. \vspace*{\baselineskip}

Best practices for the inclusion of LGBT+ people on campus have been proposed by several authors\footnote{W.~J. Blumenfeld. 1993. \emph{Making Colleges and Universities Safe for Gay and Lesbian Students: Report and Recommendations of the Massachusetts Governors Commission on Gay and Lesbian Youth}. Boston, Massachusetts.}$^,$\footnote{Rankin et al. 2010.}$^,$\footnote{{S. Windmeyer, S. Rankin, \& G. Beemyn. 2009.} {\emph{Campus Climate Index}} \link{http://www.campusprideindex.org}{(http://www.campusprideindex.org)}.}. In this document, we limit ourselves to recommendations that are particularly relevant to faculty and department chairs (as opposed to university administrators). After a brief glossary of terms, we offer both short-term and long-term department-level suggestions, then address recruitment and personnel issues.  We conclude with recommendations for university-level policies that may guide conversations with institutional administrators. Lists of useful external resources and the authors of this document are provided as appendices.
%\vspace*{\baselineskip}

\end{titlepage}