%\frontmatter
\begin{titlepage}
%\thispagestyle{empty}
\setcounter{page}{3}
\begin{center}

{\Large{\textbf{Supporting LGBT+ \\Physicists \& Astronomers:\\ Best Practices for Academic Departments}}}
\vspace*{\baselineskip}

{\large{\textbf{Introduction}}}
\vspace*{\baselineskip}
\end{center}

When physicists and astronomers discuss issues related to diversity or broadening participation in the field, the focus is typically on creating support mechanisms for women or people of color.  However, scientists who identify as lesbian, gay, bisexual, or transgender (LGBT) are also a minority within the physics and astronomy communities and can find themselves marginalized in a variety of ways.  This document aims to highlight opportunities for making academic departments more inclusive for LGBT+ students, staff, and faculty.  (We add a plus symbol to remind ourselves that not everyone fits neatly into the LGBT constructs, and some may identify differently.)\vspace*{\baselineskip}

Even if you consider your department to be a safe and welcoming space for LGBT+ students and staff, it is important to note that the campus environment is often perceived differently by different groups: A 2003 study of campus climate\footnote{S. R. Rankin. 2003. \emph{Campus Climate for Gay, Lesbian, Bisexual, and Transgender People: A National Perspective}. New York: National Gay and Lesbian Task Force Policy Institute.} found that, while 90\% of heterosexual students classified their campuses as �friendly,� 74\% of LGBT students rated the campus climate as �homophobic.� LGBT staff and faculty routinely report that their campuses are more homophobic than reported by students. A 2010 study\footnote{S.~R. Rankin, G. Weber, W. Blumenfeld, \& S. Frazer. 2010. {\em State of Higher Education for Lesbian, Gay, Bisexual and Transgender People.} Charlotte, NC: Campus Pride.} found that 23\% of LGB respondents had been harassed within the past year, with even higher rates (31--39\%) for transgender individuals. Around half of all LGB students and two-thirds of transgender students had avoided disclosing their identity to avoid harassment. \vspace*{\baselineskip}

LGBT+ youth are particularly vulnerable: they make up an estimated 40\% of homeless youth\footnote{L. E. Durso \& G. J. Gates. 2012. Serving our youth: Findings from a national survey of service providers working with lesbian, gay, bisexual, and transgender youth who are homeless or at risk of becoming homeless. Los Angeles: The Williams Institute with True Colors Fund and the Palette Fund. Available at \href{http://williamsinstitute.law.ucla.edu/wp-content/uploads/Durso-Gates-LGBT-Homeless-Youth-Survey-July-2012.pdf}{this URL.}} and are four times more likely than their straight peers to attempt suicide\footnote{CDC. 2011. Sexual identity, sex or sexual contacts, and health risk behaviors among students in Grades 9-12: Youth risk behavior surveillance. Atlanta, GA: US Department of Health and Human Services.}. Nearly half of transgender youth have seriously considered suicide and one quarter report having made a suicide attempt\footnote{A.~H. Grossman \& A.~R. D'Augelli. 2007. Transgender youth and life-threatening behaviors.  {\em Suicide and Life-Threatening Behaviors}, 37(5), 527-537.}.  This is not to say that LGBT+ students are inevitably in crisis when they enter our departments -- most arrive on campus with an astonishing amount of resilience -- but it is important to be aware of the issues that might have affected them in high school and might continue to do so throughout their university career.  \vspace*{\baselineskip}

The argument for inclusion is simple: science advances fastest when the best scientists are free to apply their intelligence and imagination to the exploration of the universe without limits and without fear.  Sometimes, the best scientists are LGBT+.  Institutions that are viewed as unfriendly to LGBT+ people quickly find themselves at a competitive disadvantage.  When LGBT+ scientists leave our departments to work at other institutions, our students, our scholarly communities, and our own research suffer.  Furthermore, a more inclusive workplace has advantages for all of us: greater flexibility to perform our work, greater support for work/life issues, and greater freedom to be ourselves. \vspace*{\baselineskip}

Research has shown that the presence of ``difference'' on campus is important to the intellectual and social development of the majority\footnote{R.~D. Reason, B.~E. Cox, B.~R.~L. Quaye, \& P.~T. Terenzini. 2010. Faculty and institutional 
factors that promote student encounters with difference in first-year courses. {\em The Review of Higher Education}, 33(3), 391-414.}$^,$\footnote{H. Smith, R. Parr, R. Woods, B. Bauer, \& T. Abraham. 2010. Five years after graduation: Undergraduate cross-group friendships and multicultural curriculum predict current attitudes and activities. {\em Journal of College Student Development}, 51(4), 385-402.}. Specifically, students who interact with people different from themselves show more developed critical and creative thinking, improved cross-cultural relationships, and increased volunteerism and civil activism. Love\footnote{P. Love. 1997. Contradiction and paradox: Attempting to change the culture of sexual orientation at a small Catholic college. {\em The Review of Higher Education}, 20(4), 381-398.} also found that students and staff who encounter contradictions in their own thinking around LGBT+ issues can develop greater levels of cognitive sophistication. The advantages of engaging with ``difference'' appear to continue beyond college and into the workforce\footnote{Smith et al.}. In other words, a diverse department can improve your students' performance, make them better capable of functioning in today's multicultural workforce, and improve the reputation of your department. \vspace*{\baselineskip}

Best practices for the inclusion of LGBT+ people on campus have been proposed by several authors\footnote{W.~J. Blumenfeld. 1993. \emph{Making Colleges and Universities Safe for Gay and Lesbian Students: Report and Recommendations of the Massachusetts Governors Commission on Gay and Lesbian Youth}. Boston, Massachusetts.}$^,$\footnote{Rankin. 2003.}$^,$\footnote{S. Windmeyer, S. Rankin, G. Beemyn. 2009. \emph{Campus Climate Index}. \href{http://www.campusprideindex.org}{campusprideindex.org}.}. In this document, we limit ourselves to recommendations that are particularly relevant to faculty and department chairs (as opposed to university administrators). After a brief glossary of terms, we offer both short-term and long-term department-level suggestions, then address recruitment and personnel issues.  We conclude with recommendations for university-level policies that may guide conversations with institutional administrators. A list of useful external resources is available at the end of the document, along with the author list.\vspace*{\baselineskip}

\end{titlepage}
